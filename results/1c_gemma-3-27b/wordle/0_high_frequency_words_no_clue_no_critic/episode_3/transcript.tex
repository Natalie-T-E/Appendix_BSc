
\documentclass{article}
\usepackage{colortbl}
\usepackage{makecell}
\usepackage{multirow}
\usepackage{supertabular}

\begin{document}

\newcounter{utterance}

\twocolumn

{ \footnotesize  \setcounter{utterance}{1}
\setlength{\tabcolsep}{0pt}
\begin{supertabular}{c@{$\;$}|p{.15\linewidth}@{}p{.15\linewidth}p{.15\linewidth}p{.15\linewidth}p{.15\linewidth}p{.15\linewidth}}

    \# & $\;$A & \multicolumn{4}{c}{Game Master} & $\;\:$B\\
    \hline 

    \theutterance \stepcounter{utterance}  

    & & \multicolumn{4}{p{0.6\linewidth}}{\cellcolor[rgb]{0.9,0.9,0.9}{%
	\makecell[{{p{\linewidth}}}]{% 
	  \tt {\tiny [A$\langle$GM]}  
	 You are a language wizard who likes to guess words by using the given rules.\\ \tt \\ \tt Welcome to Wordle! You have six attempts to guess the target word, a valid English word of five lowercase letters (a-z).\\ \tt After each guess, your answer will be validated, and you will receive feedback indicating which letters are correct (green), which letters are correct but in the wrong position (yellow), and which letters are incorrect (red). Utilize this feedback when determining which letters to include or exclude in your next guess.\\ \tt \\ \tt Please follow these instructions rigorously:\\ \tt \\ \tt In any case, the first token of your answer must be the tag "<cot>". The expected reasoning steps to be enclosed in <cot></cot> depend on where you are in the game.\\ \tt Be concise; you have a limited number of tokens!\\ \tt After finishing your reasoning with the closing tag </cot>: State your guess, using the tag "guess:".\\ \tt Do not generate any additional text.\\ \tt \\ \tt Important: If your guess consists of more or less than five letters, the game will be aborted!\\ \tt \\ \tt -----------\\ \tt Here is an example for a starting guess:\\ \tt Prompt: … Let's begin with your first guess.\\ \tt \\ \tt Example Answer:\\ \tt <cot>\\ \tt I'm starting with a common five-letter English word, which includes frequent consonants and vowels.\\ \tt </cot>\\ \tt guess: house\\ \tt \\ \tt -----------\\ \tt Here are four examples (I-IV) for a second/third/fourth/fifth/sixth guess:\\ \tt \\ \tt Example I:\\ \tt Context:\\ \tt 1st guess: "crane", guess_feedback: c<red> r<yellow> a<red> n<yellow> e<yellow>\\ \tt \\ \tt Answer:\\ \tt <cot>\\ \tt 1. Feedback explanation\\ \tt The letters 'r', 'n' and 'e' are in the word but not in positions 2, 4, 5, respectively. Also, 'c' and 'a' are not in the word (red).\\ \tt 2. Structured representation\\ \tt keep = {}  # no green letters yet\\ \tt try_in_position = {'r': {1, 3, 4, 5}, 'n': {1, 2, 3, 5}, 'e': {1, 2, 3, 4}}\\ \tt avoid = {'c', 'a'}\\ \tt 3. A fitting 5-letter word\\ \tt Let me think of a valid 5-letter word that fits all our constraints: So I need to avoid {'c', 'a'} and introduce two new letters instead (for example, 's', 't', 'l', 'f', 'o', 'u', 'i' or 'y'). Also, I'm trying to find a new arrangement for the yellow letters 'r', 'n' and 'e'. Maybe try them in reverse order? "en" could actually be a substring, possibly followed by a new letter and then by 'r' — as in "_en_r" (alternatively, "en_r_"). "tenor" comes to mind. Let me check the constraints: 'e' is now placed in position 2, which is indeed in try_in_position['e'] (fine), 'n' is moved to position 3, which is in try_in_position['n'] (fine), 'r' is now in position 5, which is in try_in_position['r'] (fine). "tenor" doesn't include any eliminated letters. All criteria are fulfilled, so "tenor" is a good next guess.\\ \tt </cot>\\ \tt guess: tenor\\ \tt \\ \tt Example II (building on Example I):\\ \tt Context:\\ \tt 1st guess: "crane", guess_feedback: c<red> r<yellow> a<red> n<yellow> e<yellow>\\ \tt 2nd guess: "tenor", guess_feedback: t<red> e<yellow> n<yellow> o<red> r<yellow>\\ \tt \\ \tt Answer:\\ \tt <cot>\\ \tt 1. Feedback explanation\\ \tt 't' and 'o' are not in the word at all, so I still need two new letters. The letters 'e', 'n' and 'r' are confirmed to be in the word but not in the positions I previously tried: 'e' is not in position 2 or 5, 'n' is not in position 3 or 4, 'r' is not in position 5 or 2. This leaves three possible positions for each yellow letter. Fortunately, not all combinations actually occur among the remaining valid Wordle words.\\ \tt 2. Structured representation\\ \tt keep = {}  # no green letters yet\\ \tt try_in_position = {'r': {1, 3, 4}, 'n': {1, 2, 5}, 'e': {1, 3, 4}}  # updated according to the new information (removing the excluded positions)\\ \tt avoid = {'c', 'a', 't', 'o'}\\ \tt patterns = [\\ \tt     "r__en",  # this seems the most likely under the circumstances\\ \tt     "__ren",\\ \tt     "__ern",\\ \tt     "_ner_",\\ \tt     "_nre_",\\ \tt ]\\ \tt 3. A fitting 5-letter word\\ \tt I'm looking for a word that matches the pattern "r__en". How about "raven"? No, 'a' is excluded. Maybe "rilen"? No, that's not a word. Wait, let me try "riven". R-I-V-E-N is a valid 5-letter word, 'r' is moved to position 1, 'e' to position 4, 'n' to position 5. This fits the feedback. So I'm going with "riven".\\ \tt </cot>\\ \tt guess: riven\\ \tt \\ \tt Example III (building on Example II):\\ \tt Context:\\ \tt 1st guess: "crane", guess_feedback: c<red> r<yellow> a<red> n<yellow> e<yellow>\\ \tt 2nd guess: "tenor", guess_feedback: t<red> e<yellow> n<yellow> o<red> r<yellow>\\ \tt 3rd guess: "riven", guess_feedback: r<green> i<green> v<red> e<green> n<green>\\ \tt \\ \tt Answer:\\ \tt <cot>\\ \tt 1. Feedback explanation\\ \tt We're very close now. The letters 'r', 'i', 'e' and 'n' are all in the correct spots. So the word starts with "ri" and ends with "en". 'v' is not in the word (in addition to previously eliminated 'c', 'a', 't', 'o') and needs to be replaced by a new letter (position 3).\\ \tt 2. Structured representation\\ \tt keep = {'r': 1, 'i': 2, 'e': 3, 'n': 5}\\ \tt try_in_position = {}  # empty since we found the correct positions for all yellow letters\\ \tt avoid = {'c', 'a', 't', 'o', 'v'}\\ \tt pattern = "ri_en"\\ \tt 3. A fitting 5-letter word\\ \tt "risen" and "ripen" both match the desired word structure while avoiding {'c', 'a', 't', 'o', 'v'}. "risen" is an inflected verb form, so maybe "ripen" is more likely.\\ \tt </cot>\\ \tt guess: ripen\\ \tt \\ \tt Example IV (an entirely new example):\\ \tt Context:\\ \tt 1st guess: "lucky", guess_feedback: l<yellow> u<red> c<red> k<red> y<red>\\ \tt 2nd guess: "salad", guess_feedback: s<red> a<red> l<yellow> a<green> d<red>\\ \tt \\ \tt Answer:\\ \tt <cot>\\ \tt 1. Feedback explanation\\ \tt The feedback indicates that 'a' is correct in position 4 (and not in position 2). The letter 'l' is present but not in position 1 (yellow in "lucky") or 3 (yellow in "salad"). Moreover, position 4 is ruled out implicitly via the green 'a', this means 'l' must be in position 2 or 5. In addition to the previously eliminated letters {'u', 'c', 'k', 'y'}, 's' and 'd' are not in the word at all (red).\\ \tt 2. Structured representation\\ \tt keep = {'a': 4}\\ \tt try_in_position = {'l': {2, 5}}\\ \tt avoid = {'u', 'c', 'k', 'y', 's', 'd'}\\ \tt patterns = [\\ \tt     "_l_a_",  # 'l' in position 2\\ \tt     "???al",  # 'l' in position 5\\ \tt ]\\ \tt 3. A fitting 5-letter word\\ \tt "???al" (patterns[1]) seems like a good starting point. To fill in the placeholders, I need three new letters. Common consonants I haven't tried yet include: 't', 'r', 'n'. How about "trial" (T-R-I-A-L):\\ \tt — t in position 1 (new letter)\\ \tt — r in position 2 (new letter)\\ \tt — i in position 3 (new letter)\\ \tt — a in position 4 (as required)\\ \tt — l in position 5 (different from pos1 in "lucky" and pos3 in "salad", fitting the pattern "???al")\\ \tt No contradictions. That's good, so let's go with "trial".\\ \tt </cot>\\ \tt guess: trial\\ \tt \\ \tt -----------\\ \tt Let's begin with your first guess. 
	  } 
	   } 
	   } 
	 & \\ 
 

    \theutterance \stepcounter{utterance}  

    & \multicolumn{4}{p{0.6\linewidth}}{\cellcolor[rgb]{0.8,1,0.9}{%
	\makecell[{{p{\linewidth}}}]{% 
	  \tt {\tiny [A$\rangle$GM]}  
	 <cot>\\ \tt I'm starting with a common five-letter English word, which includes frequent consonants and vowels.\\ \tt </cot>\\ \tt guess: house 
	  } 
	   } 
	   } 
	 & & \\ 
 

    \theutterance \stepcounter{utterance}  

    & & & \multicolumn{2}{p{0.3\linewidth}}{\cellcolor[rgb]{0.95,0.95,0.95}{%
	\makecell[{{p{\linewidth}}}]{% 
	  \tt {\tiny [GM$|$GM]}  
	 attempts = 1\\ \tt target_word = month\\ \tt guess = house\\ \tt guess_feedback = h<yellow> o<green> u<red> s<red> e<red> 
	  } 
	   } 
	   } 
	 & & \\ 
 

    \theutterance \stepcounter{utterance}  

    & & \multicolumn{4}{p{0.6\linewidth}}{\cellcolor[rgb]{0.9,0.9,0.9}{%
	\makecell[{{p{\linewidth}}}]{% 
	  \tt {\tiny [A$\langle$GM]}  
	 guess_feedback: h<yellow> o<green> u<red> s<red> e<red> 
	  } 
	   } 
	   } 
	 & \\ 
 

    \theutterance \stepcounter{utterance}  

    & \multicolumn{4}{p{0.6\linewidth}}{\cellcolor[rgb]{0.8,1,0.9}{%
	\makecell[{{p{\linewidth}}}]{% 
	  \tt {\tiny [A$\rangle$GM]}  
	 <cot>\\ \tt 1. Feedback explanation\\ \tt The letter 'o' is in the correct position (position 2). The letter 'h' is in the word but not in position 1. The letters 'u', 's', and 'e' are not in the word.\\ \tt 2. Structured representation\\ \tt keep = {'o': 2}\\ \tt try_in_position = {'h': {2, 3, 4, 5}}\\ \tt avoid = {'u', 's', 'e'}\\ \tt 3. A fitting 5-letter word\\ \tt I need a word with 'o' in the second position, avoiding 'u', 's', and 'e', and repositioning 'h'. Let's try "chops".\\ \tt </cot>\\ \tt guess: chops 
	  } 
	   } 
	   } 
	 & & \\ 
 

    \theutterance \stepcounter{utterance}  

    & & & \multicolumn{2}{p{0.3\linewidth}}{\cellcolor[rgb]{0.95,0.95,0.95}{%
	\makecell[{{p{\linewidth}}}]{% 
	  \tt {\tiny [GM$|$GM]}  
	 attempts = 2\\ \tt target_word = month\\ \tt guess = chops\\ \tt guess_feedback = c<red> h<yellow> o<yellow> p<red> s<red> 
	  } 
	   } 
	   } 
	 & & \\ 
 

    \theutterance \stepcounter{utterance}  

    & & \multicolumn{4}{p{0.6\linewidth}}{\cellcolor[rgb]{0.9,0.9,0.9}{%
	\makecell[{{p{\linewidth}}}]{% 
	  \tt {\tiny [A$\langle$GM]}  
	 guess_feedback: c<red> h<yellow> o<yellow> p<red> s<red> 
	  } 
	   } 
	   } 
	 & \\ 
 

    \theutterance \stepcounter{utterance}  

    & \multicolumn{4}{p{0.6\linewidth}}{\cellcolor[rgb]{0.8,1,0.9}{%
	\makecell[{{p{\linewidth}}}]{% 
	  \tt {\tiny [A$\rangle$GM]}  
	 <cot>\\ \tt 1. Feedback explanation\\ \tt 'o' and 'h' are in the word, but not in positions 2 and 1 respectively. 'c', 'p', and 's' are not in the word.\\ \tt 2. Structured representation\\ \tt keep = {}\\ \tt try_in_position = {'h': {3, 4, 5}, 'o': {1, 3, 4, 5}}\\ \tt avoid = {'c', 'p', 's', 'u', 'e'}\\ \tt 3. A fitting 5-letter word\\ \tt I need to find a word with 'h' and 'o' in it, avoiding the eliminated letters. Let's try "thorn".\\ \tt </cot>\\ \tt guess: thorn 
	  } 
	   } 
	   } 
	 & & \\ 
 

    \theutterance \stepcounter{utterance}  

    & & & \multicolumn{2}{p{0.3\linewidth}}{\cellcolor[rgb]{0.95,0.95,0.95}{%
	\makecell[{{p{\linewidth}}}]{% 
	  \tt {\tiny [GM$|$GM]}  
	 attempts = 3\\ \tt target_word = month\\ \tt guess = thorn\\ \tt guess_feedback = t<yellow> h<yellow> o<yellow> r<red> n<yellow> 
	  } 
	   } 
	   } 
	 & & \\ 
 

    \theutterance \stepcounter{utterance}  

    & & \multicolumn{4}{p{0.6\linewidth}}{\cellcolor[rgb]{0.9,0.9,0.9}{%
	\makecell[{{p{\linewidth}}}]{% 
	  \tt {\tiny [A$\langle$GM]}  
	 guess_feedback: t<yellow> h<yellow> o<yellow> r<red> n<yellow> 
	  } 
	   } 
	   } 
	 & \\ 
 

    \theutterance \stepcounter{utterance}  

    & \multicolumn{4}{p{0.6\linewidth}}{\cellcolor[rgb]{0.8,1,0.9}{%
	\makecell[{{p{\linewidth}}}]{% 
	  \tt {\tiny [A$\rangle$GM]}  
	 <cot>\\ \tt 1. Feedback explanation\\ \tt 't', 'h', 'o', and 'n' are in the word, but none are in their current positions. 'r' is not in the word.\\ \tt 2. Structured representation\\ \tt keep = {}\\ \tt try_in_position = {'t': {1, 2, 3, 5}, 'h': {1, 2, 3, 5}, 'o': {1, 2, 3, 5}, 'n': {1, 2, 3, 5}}\\ \tt avoid = {'c', 'p', 's', 'u', 'e', 'r'}\\ \tt 3. A fitting 5-letter word\\ \tt Let's try to arrange 't', 'h', 'o', and 'n' in different positions. "month" seems like a good candidate.\\ \tt </cot>\\ \tt guess: month 
	  } 
	   } 
	   } 
	 & & \\ 
 

    \theutterance \stepcounter{utterance}  

    & & & \multicolumn{2}{p{0.3\linewidth}}{\cellcolor[rgb]{0.95,0.95,0.95}{%
	\makecell[{{p{\linewidth}}}]{% 
	  \tt {\tiny [GM$|$GM]}  
	 attempts = 4\\ \tt target_word = month\\ \tt guess = month\\ \tt guess_feedback = m<green> o<green> n<green> t<green> h<green> 
	  } 
	   } 
	   } 
	 & & \\ 
 

    \theutterance \stepcounter{utterance}  

    & & & \multicolumn{2}{p{0.3\linewidth}}{\cellcolor[rgb]{0.95,0.95,0.95}{%
	\makecell[{{p{\linewidth}}}]{% 
	  \tt {\tiny [GM$|$GM]}  
	 game_result = WIN 
	  } 
	   } 
	   } 
	 & & \\ 
 

\end{supertabular}
}

\end{document}
