
\documentclass{article}
\usepackage{colortbl}
\usepackage{makecell}
\usepackage{multirow}
\usepackage{supertabular}

\begin{document}

\newcounter{utterance}

\twocolumn

{ \footnotesize  \setcounter{utterance}{1}
\setlength{\tabcolsep}{0pt}
\begin{supertabular}{c@{$\;$}|p{.15\linewidth}@{}p{.15\linewidth}p{.15\linewidth}p{.15\linewidth}p{.15\linewidth}p{.15\linewidth}}

    \# & $\;$A & \multicolumn{4}{c}{Game Master} & $\;\:$B\\
    \hline 

    \theutterance \stepcounter{utterance}  

    & & \multicolumn{4}{p{0.6\linewidth}}{\cellcolor[rgb]{0.9,0.9,0.9}{%
	\makecell[{{p{\linewidth}}}]{% 
	  \tt {\tiny [A$\langle$GM]}  
	 You are a language wizard who likes to guess words by using the given rules.\\ \tt \\ \tt Welcome to Wordle! You have six attempts to guess the target word, a valid English word of five lowercase letters (a-z).\\ \tt After each guess, your answer will be validated, and you will receive feedback indicating which letters are correct (green), which letters are correct but in the wrong position (yellow), and which letters are incorrect (red). Utilize this feedback when determining which letters to include or exclude in your next guess.\\ \tt \\ \tt Please follow these instructions rigorously:\\ \tt \\ \tt In any case, the first token of your answer must be the tag "<cot>". The expected reasoning steps to be enclosed in <cot></cot> depend on where you are in the game.\\ \tt Be concise; you have a limited number of tokens!\\ \tt After finishing your reasoning with the closing tag </cot>: State your guess, using the tag "guess:".\\ \tt Do not generate any additional text.\\ \tt \\ \tt Important: If your guess consists of more or less than five letters, the game will be aborted!\\ \tt \\ \tt -----------\\ \tt Here is an example for a starting guess:\\ \tt Prompt: … Let's begin with your first guess.\\ \tt \\ \tt Example Answer:\\ \tt <cot>\\ \tt I'm starting with a common five-letter English word, which includes frequent consonants and vowels.\\ \tt </cot>\\ \tt guess: house\\ \tt \\ \tt -----------\\ \tt Here are two examples for a second/third/fourth/fifth/sixth guess:\\ \tt \\ \tt Example I:\\ \tt Context:\\ \tt 1st guess: "crane", guess_feedback: c<red> r<yellow> a<red> n<yellow> e<yellow>\\ \tt \\ \tt Answer:\\ \tt <cot>\\ \tt 1. Feedback explanation\\ \tt The letters 'r', 'n' and 'e' are in the word but not in positions 2, 4, 5, respectively. Also, 'c' and 'a' are not in the word (red).\\ \tt 2. A fitting 5-letter word\\ \tt Let me think of a valid 5-letter word that fits all our constraints: So I need to avoid 'c' and 'a' and introduce two new letters instead (for example, 's', 't', 'l', 'f', 'o', 'u', 'i' or 'y'). Also, I'm trying to find a new arrangement for the yellow letters 'r', 'n' and 'e'. Maybe try them in reverse order? "en" could actually be a substring, possibly followed by a new letter and then by 'r' — as in "_en_r" (alternatively, "en_r_"). "tenor" comes to mind. Let me check the constraints: 'e' is now placed in position 2, which is indeed different from position 5 in "crane" (fine), 'n' was incorrect in position 4 and is moved to position 3 (fine), 'r' was incorrect in position 2 and is now in position 5 (fine). "tenor" doesn't include any eliminated letters. All criteria are fulfilled, so "tenor" is a good next guess.\\ \tt </cot>\\ \tt guess: tenor\\ \tt \\ \tt Example II:\\ \tt Context:\\ \tt 1st guess: "lucky", guess_feedback: l<yellow> u<red> c<red> k<red> y<red>\\ \tt 2nd guess: "salad", guess_feedback: s<red> a<red> l<yellow> a<green> d<red>\\ \tt \\ \tt Answer:\\ \tt <cot>\\ \tt 1. Feedback explanation\\ \tt The feedback indicates that 'a' is correct in position 4 (and not in position 2). The letter 'l' is present but not in position 1 (yellow in "lucky") or 3 (yellow in "salad"). Moreover, position 4 is ruled out implicitly via the green 'a', this means 'l' must be in position 2 or 5. In addition to the previously eliminated letters ('u', 'c', 'k', 'y'), 's' and 'd' are not in the word at all (red).\\ \tt 2. A fitting 5-letter word\\ \tt I need a word that fits the pattern "_l_a_" with 'l' in position 2, alternatively "???al", placing 'l' in position 5. "???al" seems like a good starting point. To fill in the placeholders, I need three new letters. Common consonants I haven't tried yet include: 't', 'r', 'n'. How about "trial" (T-R-I-A-L):\\ \tt — t in position 1 (new letter)\\ \tt — r in position 2 (new letter)\\ \tt — i in position 3 (new letter)\\ \tt — a in position 4 (as required)\\ \tt — l in position 5 (different from pos1 in "lucky" and pos3 in "salad", fitting the pattern "???al")\\ \tt No contradictions. That's good, so let's go with "trial".\\ \tt </cot>\\ \tt guess: trial\\ \tt \\ \tt -----------\\ \tt Let's begin with your first guess. 
	  } 
	   } 
	   } 
	 & \\ 
 

    \theutterance \stepcounter{utterance}  

    & \multicolumn{4}{p{0.6\linewidth}}{\cellcolor[rgb]{0.8,1,0.9}{%
	\makecell[{{p{\linewidth}}}]{% 
	  \tt {\tiny [A$\rangle$GM]}  
	 <cot>\\ \tt I'm starting with a common five-letter English word, which includes frequent consonants and vowels. A good initial guess could be a word that has a balance of vowels and consonants, such as "house".\\ \tt </cot>\\ \tt guess: house 
	  } 
	   } 
	   } 
	 & & \\ 
 

    \theutterance \stepcounter{utterance}  

    & & & \multicolumn{2}{p{0.3\linewidth}}{\cellcolor[rgb]{0.95,0.95,0.95}{%
	\makecell[{{p{\linewidth}}}]{% 
	  \tt {\tiny [GM$|$GM]}  
	 attempts = 1\\ \tt target_word = rally\\ \tt guess = house\\ \tt guess_feedback = h<red> o<red> u<red> s<red> e<red> 
	  } 
	   } 
	   } 
	 & & \\ 
 

    \theutterance \stepcounter{utterance}  

    & & \multicolumn{4}{p{0.6\linewidth}}{\cellcolor[rgb]{0.9,0.9,0.9}{%
	\makecell[{{p{\linewidth}}}]{% 
	  \tt {\tiny [A$\langle$GM]}  
	 guess_feedback: h<red> o<red> u<red> s<red> e<red> 
	  } 
	   } 
	   } 
	 & \\ 
 

    \theutterance \stepcounter{utterance}  

    & \multicolumn{4}{p{0.6\linewidth}}{\cellcolor[rgb]{0.8,1,0.9}{%
	\makecell[{{p{\linewidth}}}]{% 
	  \tt {\tiny [A$\rangle$GM]}  
	 <cot>\\ \tt The feedback indicates that none of the letters in "house" are in the target word. This means I need to start fresh with a new set of letters. A new approach could be to try a word with a different set of common consonants and vowels, such as "bread" or "cloud", but let's try "bread" first.\\ \tt </cot>\\ \tt guess: bread 
	  } 
	   } 
	   } 
	 & & \\ 
 

    \theutterance \stepcounter{utterance}  

    & & & \multicolumn{2}{p{0.3\linewidth}}{\cellcolor[rgb]{0.95,0.95,0.95}{%
	\makecell[{{p{\linewidth}}}]{% 
	  \tt {\tiny [GM$|$GM]}  
	 attempts = 2\\ \tt target_word = rally\\ \tt guess = bread\\ \tt guess_feedback = b<red> r<yellow> e<red> a<yellow> d<red> 
	  } 
	   } 
	   } 
	 & & \\ 
 

    \theutterance \stepcounter{utterance}  

    & & \multicolumn{4}{p{0.6\linewidth}}{\cellcolor[rgb]{0.9,0.9,0.9}{%
	\makecell[{{p{\linewidth}}}]{% 
	  \tt {\tiny [A$\langle$GM]}  
	 guess_feedback: b<red> r<yellow> e<red> a<yellow> d<red> 
	  } 
	   } 
	   } 
	 & \\ 
 

    \theutterance \stepcounter{utterance}  

    & \multicolumn{4}{p{0.6\linewidth}}{\cellcolor[rgb]{0.8,1,0.9}{%
	\makecell[{{p{\linewidth}}}]{% 
	  \tt {\tiny [A$\rangle$GM]}  
	 <cot>\\ \tt The feedback shows that 'r' and 'a' are in the word, but not in the positions they occupied in "bread". Since 'r' was in position 2 and 'a' was in position 4, my next guess should try them in different positions. A possible arrangement could be to place 'r' at the end and 'a' near the beginning. Let's try a word like "taper" or "cared", but since 'b', 'e', and 'd' are eliminated, "taper" seems like a good option, but 'e' is also eliminated, so let's think of another word. How about "traps" or "wrack", but 'b' is eliminated and 'w' and 'c' and 'k' are new, "wrack" could work if 'w' is a good fit.\\ \tt </cot>\\ \tt guess: wrack 
	  } 
	   } 
	   } 
	 & & \\ 
 

    \theutterance \stepcounter{utterance}  

    & & & \multicolumn{2}{p{0.3\linewidth}}{\cellcolor[rgb]{0.95,0.95,0.95}{%
	\makecell[{{p{\linewidth}}}]{% 
	  \tt {\tiny [GM$|$GM]}  
	 attempts = 3\\ \tt target_word = rally\\ \tt guess = wrack\\ \tt guess_feedback = w<red> r<yellow> a<yellow> c<red> k<red> 
	  } 
	   } 
	   } 
	 & & \\ 
 

    \theutterance \stepcounter{utterance}  

    & & \multicolumn{4}{p{0.6\linewidth}}{\cellcolor[rgb]{0.9,0.9,0.9}{%
	\makecell[{{p{\linewidth}}}]{% 
	  \tt {\tiny [A$\langle$GM]}  
	 guess_feedback: w<red> r<yellow> a<yellow> c<red> k<red> 
	  } 
	   } 
	   } 
	 & \\ 
 

    \theutterance \stepcounter{utterance}  

    & \multicolumn{4}{p{0.6\linewidth}}{\cellcolor[rgb]{0.8,1,0.9}{%
	\makecell[{{p{\linewidth}}}]{% 
	  \tt {\tiny [A$\rangle$GM]}  
	 <cot>\\ \tt The feedback indicates that 'r' and 'a' are still in the word, but not in the positions they occupied in "wrack". Since 'w', 'c', and 'k' are eliminated, my next guess should try to utilize 'r' and 'a' in different positions, and introduce new letters. Considering the previous attempts, 'r' has been in positions 2 and 2 again, and 'a' has been in positions 4 and 3, let's try placing 'r' in position 5 and 'a' in position 1 or 2, and fill in the rest with new letters. A possible word could be "arise" or "alarm" but 'e' and 'l' and 'm' are new, "arise" could work.\\ \tt </cot>\\ \tt guess: arise 
	  } 
	   } 
	   } 
	 & & \\ 
 

    \theutterance \stepcounter{utterance}  

    & & & \multicolumn{2}{p{0.3\linewidth}}{\cellcolor[rgb]{0.95,0.95,0.95}{%
	\makecell[{{p{\linewidth}}}]{% 
	  \tt {\tiny [GM$|$GM]}  
	 attempts = 4\\ \tt target_word = rally\\ \tt guess = arise\\ \tt guess_feedback = a<yellow> r<yellow> i<red> s<red> e<red> 
	  } 
	   } 
	   } 
	 & & \\ 
 

    \theutterance \stepcounter{utterance}  

    & & \multicolumn{4}{p{0.6\linewidth}}{\cellcolor[rgb]{0.9,0.9,0.9}{%
	\makecell[{{p{\linewidth}}}]{% 
	  \tt {\tiny [A$\langle$GM]}  
	 guess_feedback: a<yellow> r<yellow> i<red> s<red> e<red> 
	  } 
	   } 
	   } 
	 & \\ 
 

    \theutterance \stepcounter{utterance}  

    & \multicolumn{4}{p{0.6\linewidth}}{\cellcolor[rgb]{0.8,1,0.9}{%
	\makecell[{{p{\linewidth}}}]{% 
	  \tt {\tiny [A$\rangle$GM]}  
	 <cot>\\ \tt The feedback shows that 'a' and 'r' are in the word, but not in the positions they occupied in "arise". Since 'i', 's', and 'e' are eliminated, my next guess should try to utilize 'a' and 'r' in different positions. Considering the previous attempts, 'a' has been in positions 4, 3, and 1, and 'r' has been in positions 2, 2, and 2, let's try placing 'a' in position 2 or 3 and 'r' in position 4 or 5, and fill in the rest with new letters. A possible word could be "tarns" but 'i' and 's' are eliminated, or "cart" and add two new letters, "carat" or "carts" but 's' is eliminated, how about "carro" no, "coral" no, "carn" and add two letters, let's think of a word that has 'a' in position 2 and 'r' in position 5, like "barns" but 'b' is not tried yet and 's' is eliminated, or "garn" and add two letters.\\ \tt </cot>\\ \tt guess: garns 
	  } 
	   } 
	   } 
	 & & \\ 
 

    \theutterance \stepcounter{utterance}  

    & & & \multicolumn{2}{p{0.3\linewidth}}{\cellcolor[rgb]{0.95,0.95,0.95}{%
	\makecell[{{p{\linewidth}}}]{% 
	  \tt {\tiny [GM$|$GM]}  
	 Guesser Error: NOT_VALID_WORD_FOR_GAME while parsing Player 1's (model: llama-3.3-70b) response 
	  } 
	   } 
	   } 
	 & & \\ 
 

    \theutterance \stepcounter{utterance}  

    & & \multicolumn{4}{p{0.6\linewidth}}{\cellcolor[rgb]{0.9,0.9,0.9}{%
	\makecell[{{p{\linewidth}}}]{% 
	  \tt {\tiny [A$\langle$GM]}  
	 Your guess is not a valid word for this game. Please try again. 
	  } 
	   } 
	   } 
	 & \\ 
 

    \theutterance \stepcounter{utterance}  

    & & \multicolumn{4}{p{0.6\linewidth}}{\cellcolor[rgb]{0.9,0.9,0.9}{%
	\makecell[{{p{\linewidth}}}]{% 
	  \tt {\tiny [A$\langle$GM]}  
	 Your guess is not a valid word for this game. Please try again. 
	  } 
	   } 
	   } 
	 & \\ 
 

    \theutterance \stepcounter{utterance}  

    & \multicolumn{4}{p{0.6\linewidth}}{\cellcolor[rgb]{0.8,1,0.9}{%
	\makecell[{{p{\linewidth}}}]{% 
	  \tt {\tiny [A$\rangle$GM]}  
	 <cot>\\ \tt The feedback shows that 'a' and 'r' are in the word, but not in the positions they occupied in "arise". Since 'i', 's', and 'e' are eliminated, my next guess should try to utilize 'a' and 'r' in different positions. Considering the previous attempts, 'a' has been in positions 4, 3, and 1, and 'r' has been in positions 2, 2, and 2, let's try placing 'a' in position 2 or 3 and 'r' in position 4 or 5, and fill in the rest with new letters. A possible word could be "baron".\\ \tt </cot>\\ \tt guess: baron 
	  } 
	   } 
	   } 
	 & & \\ 
 

    \theutterance \stepcounter{utterance}  

    & & & \multicolumn{2}{p{0.3\linewidth}}{\cellcolor[rgb]{0.95,0.95,0.95}{%
	\makecell[{{p{\linewidth}}}]{% 
	  \tt {\tiny [GM$|$GM]}  
	 attempts = 5\\ \tt target_word = rally\\ \tt guess = baron\\ \tt guess_feedback = b<red> a<green> r<yellow> o<red> n<red> 
	  } 
	   } 
	   } 
	 & & \\ 
 

    \theutterance \stepcounter{utterance}  

    & & \multicolumn{4}{p{0.6\linewidth}}{\cellcolor[rgb]{0.9,0.9,0.9}{%
	\makecell[{{p{\linewidth}}}]{% 
	  \tt {\tiny [A$\langle$GM]}  
	 guess_feedback: b<red> a<green> r<yellow> o<red> n<red> 
	  } 
	   } 
	   } 
	 & \\ 
 

    \theutterance \stepcounter{utterance}  

    & \multicolumn{4}{p{0.6\linewidth}}{\cellcolor[rgb]{0.8,1,0.9}{%
	\makecell[{{p{\linewidth}}}]{% 
	  \tt {\tiny [A$\rangle$GM]}  
	 <cot>\\ \tt The feedback indicates that 'a' is in the correct position, which is position 2, and 'r' is in the word but not in the correct position. Since 'b', 'o', and 'n' are eliminated, my next guess should try to utilize 'a' in position 2 and 'r' in a different position. Considering the previous attempts, let's try placing 'r' in position 5, and fill in the rest with new letters. A possible word could be "taper" but 'e' is eliminated, or "tarps" but 's' is eliminated, how about "tarry" but 'y' is new, or "larky" no, "mark" and add two letters, let's think of a word that has 'a' in position 2 and 'r' in position 5, like "card" and add two letters, "cards" no, "cared" no, "carry" or "cart" and add two letters, "carts" no.\\ \tt </cot>\\ \tt guess: carry 
	  } 
	   } 
	   } 
	 & & \\ 
 

    \theutterance \stepcounter{utterance}  

    & & & \multicolumn{2}{p{0.3\linewidth}}{\cellcolor[rgb]{0.95,0.95,0.95}{%
	\makecell[{{p{\linewidth}}}]{% 
	  \tt {\tiny [GM$|$GM]}  
	 attempts = 6\\ \tt target_word = rally\\ \tt guess = carry\\ \tt guess_feedback = c<red> a<green> r<yellow> r<red> y<green> 
	  } 
	   } 
	   } 
	 & & \\ 
 

    \theutterance \stepcounter{utterance}  

    & & & \multicolumn{2}{p{0.3\linewidth}}{\cellcolor[rgb]{0.95,0.95,0.95}{%
	\makecell[{{p{\linewidth}}}]{% 
	  \tt {\tiny [GM$|$GM]}  
	 game_result = LOSS 
	  } 
	   } 
	   } 
	 & & \\ 
 

\end{supertabular}
}

\end{document}
