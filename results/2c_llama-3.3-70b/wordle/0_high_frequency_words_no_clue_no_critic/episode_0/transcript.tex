
\documentclass{article}
\usepackage{colortbl}
\usepackage{makecell}
\usepackage{multirow}
\usepackage{supertabular}

\begin{document}

\newcounter{utterance}

\twocolumn

{ \footnotesize  \setcounter{utterance}{1}
\setlength{\tabcolsep}{0pt}
\begin{supertabular}{c@{$\;$}|p{.15\linewidth}@{}p{.15\linewidth}p{.15\linewidth}p{.15\linewidth}p{.15\linewidth}p{.15\linewidth}}

    \# & $\;$A & \multicolumn{4}{c}{Game Master} & $\;\:$B\\
    \hline 

    \theutterance \stepcounter{utterance}  

    & & \multicolumn{4}{p{0.6\linewidth}}{\cellcolor[rgb]{0.9,0.9,0.9}{%
	\makecell[{{p{\linewidth}}}]{% 
	  \tt {\tiny [A$\langle$GM]}  
	 You are a language wizard who likes to guess words by using the given rules.\\ \tt \\ \tt Welcome to Wordle! You have six attempts to guess the target word, a valid English word of five lowercase letters (a-z).\\ \tt After each guess, your answer will be validated, and you will receive feedback indicating which letters are correct (green), which letters are correct but in the wrong position (yellow), and which letters are incorrect (red). Utilize this feedback when determining which letters to include or exclude in your next guess.\\ \tt \\ \tt Please follow these instructions rigorously:\\ \tt \\ \tt In any case, the first token of your answer must be the tag "<cot>". The expected reasoning steps to be enclosed in <cot></cot> depend on where you are in the game.\\ \tt Be concise; you have a limited number of tokens!\\ \tt After finishing your reasoning with the closing tag </cot>: State your guess, using the tag "guess:".\\ \tt Do not generate any additional text.\\ \tt \\ \tt Important: If your guess consists of more or less than five letters, the game will be aborted!\\ \tt \\ \tt -----------\\ \tt Here is an example for a starting guess:\\ \tt Prompt: … Let's begin with your first guess.\\ \tt \\ \tt Example Answer:\\ \tt <cot>\\ \tt I'm starting with a common five-letter English word, which includes frequent consonants and vowels.\\ \tt </cot>\\ \tt guess: house\\ \tt \\ \tt -----------\\ \tt Here are four examples (I-IV) for a second/third/fourth/fifth/sixth guess:\\ \tt \\ \tt Example I:\\ \tt Context:\\ \tt 1st guess: "crane", guess_feedback: c<red> r<yellow> a<red> n<yellow> e<yellow>\\ \tt \\ \tt Answer:\\ \tt <cot>\\ \tt 1. Feedback explanation\\ \tt The letters 'r', 'n' and 'e' are in the word but not in positions 2, 4, 5, respectively. Also, 'c' and 'a' are not in the word (red).\\ \tt 2. A fitting 5-letter word\\ \tt Let me think of a valid 5-letter word that fits all our constraints: So I need to avoid 'c' and 'a' and introduce two new letters instead (for example, 's', 't', 'l', 'f', 'o', 'u', 'i' or 'y'). Also, I'm trying to find a new arrangement for the yellow letters 'r', 'n' and 'e'. Maybe try them in reverse order? "en" could actually be a substring, possibly followed by a new letter and then by 'r' — as in "_en_r" (alternatively, "en_r_"). "tenor" comes to mind. Let me check the constraints: 'e' is now placed in position 2, which is indeed different from position 5 in "crane" (fine), 'n' was incorrect in position 4 and is moved to position 3 (fine), 'r' was incorrect in position 2 and is now in position 5 (fine). "tenor" doesn't include any eliminated letters. All criteria are fulfilled, so "tenor" is a good next guess.\\ \tt </cot>\\ \tt guess: tenor\\ \tt \\ \tt Example II (building on Example I):\\ \tt Context:\\ \tt 1st guess: "crane", guess_feedback: c<red> r<yellow> a<red> n<yellow> e<yellow>\\ \tt 2nd guess: "tenor", guess_feedback: t<red> e<yellow> n<yellow> o<red> r<yellow>\\ \tt \\ \tt Answer:\\ \tt <cot>\\ \tt 1. Feedback explanation\\ \tt 't' and 'o' are not in the word at all, so I still need two new letters. The letters 'e', 'n' and 'r' are confirmed to be in the word but not in the positions I previously tried: 'e' is not in position 2 or 5, 'n' is not in position 3 or 4, 'r' is not in position 5 or 2. This leaves three possible positions for each yellow letter. Fortunately, not all combinations actually occur among the remaining valid Wordle words.\\ \tt 2. A fitting 5-letter word\\ \tt Let me check "tavern": "tavern" is T-A-V-E-R-N which is 6 letters (unacceptable). Plus, 'a' is excluded. Hmm, this is challenging. Wait, let me try "riven". R-I-V-E-N is a valid 5-letter word, 'r' is moved to position 1, 'e' to position 4, 'n' to position 5. This fits the feedback. So I'm going with "riven".\\ \tt </cot>\\ \tt guess: riven\\ \tt \\ \tt Example III (building on Example II):\\ \tt Context:\\ \tt 1st guess: "crane", guess_feedback: c<red> r<yellow> a<red> n<yellow> e<yellow>\\ \tt 2nd guess: "tenor", guess_feedback: t<red> e<yellow> n<yellow> o<red> r<yellow>\\ \tt 3rd guess: "riven", guess_feedback: r<green> i<green> v<red> e<green> n<green>\\ \tt \\ \tt Answer:\\ \tt <cot>\\ \tt 1. Feedback explanation\\ \tt We're very close now. The letters 'r', 'i', 'e' and 'n' are all in the correct spots. So the word starts with "ri" and ends with "en". 'v' is not in the word (in addition to previously eliminated 'c', 'a', 't', 'o') and needs to be replaced by a new letter (position 3).\\ \tt 2. A fitting 5-letter word\\ \tt "risen" and "ripen" both match the desired word structure while avoiding 'c', 'a', 't', 'o', 'v'. "risen" is an inflected verb form, so maybe "ripen" is more likely.\\ \tt </cot>\\ \tt guess: ripen\\ \tt \\ \tt Example IV (an entirely new example):\\ \tt Context:\\ \tt 1st guess: "lucky", guess_feedback: l<yellow> u<red> c<red> k<red> y<red>\\ \tt 2nd guess: "salad", guess_feedback: s<red> a<red> l<yellow> a<green> d<red>\\ \tt \\ \tt Answer:\\ \tt <cot>\\ \tt 1. Feedback explanation\\ \tt The feedback indicates that 'a' is correct in position 4 (and not in position 2). The letter 'l' is present but not in position 1 (yellow in "lucky") or 3 (yellow in "salad"). Moreover, position 4 is ruled out implicitly via the green 'a', this means 'l' must be in position 2 or 5. In addition to the previously eliminated letters ('u', 'c', 'k', 'y'), 's' and 'd' are not in the word at all (red).\\ \tt 2. A fitting 5-letter word\\ \tt I need a word that fits the pattern "_l_a_" with 'l' in position 2, alternatively "???al", placing 'l' in position 5. "???al" seems like a good starting point. To fill in the placeholders, I need three new letters. Common consonants I haven't tried yet include: 't', 'r', 'n'. How about "trial" (T-R-I-A-L):\\ \tt — t in position 1 (new letter)\\ \tt — r in position 2 (new letter)\\ \tt — i in position 3 (new letter)\\ \tt — a in position 4 (as required)\\ \tt — l in position 5 (different from pos1 in "lucky" and pos3 in "salad", fitting the pattern "???al")\\ \tt No contradictions. That's good, so let's go with "trial".\\ \tt </cot>\\ \tt guess: trial\\ \tt \\ \tt -----------\\ \tt Let's begin with your first guess. 
	  } 
	   } 
	   } 
	 & \\ 
 

    \theutterance \stepcounter{utterance}  

    & \multicolumn{4}{p{0.6\linewidth}}{\cellcolor[rgb]{0.8,1,0.9}{%
	\makecell[{{p{\linewidth}}}]{% 
	  \tt {\tiny [A$\rangle$GM]}  
	 <cot>\\ \tt I'm starting with a common five-letter English word, which includes frequent consonants and vowels. A good starting point could be a word with a high probability of appearing in the language, such as "house".\\ \tt </cot>\\ \tt guess: house 
	  } 
	   } 
	   } 
	 & & \\ 
 

    \theutterance \stepcounter{utterance}  

    & & & \multicolumn{2}{p{0.3\linewidth}}{\cellcolor[rgb]{0.95,0.95,0.95}{%
	\makecell[{{p{\linewidth}}}]{% 
	  \tt {\tiny [GM$|$GM]}  
	 attempts = 1\\ \tt target_word = party\\ \tt guess = house\\ \tt guess_feedback = h<red> o<red> u<red> s<red> e<red> 
	  } 
	   } 
	   } 
	 & & \\ 
 

    \theutterance \stepcounter{utterance}  

    & & \multicolumn{4}{p{0.6\linewidth}}{\cellcolor[rgb]{0.9,0.9,0.9}{%
	\makecell[{{p{\linewidth}}}]{% 
	  \tt {\tiny [A$\langle$GM]}  
	 guess_feedback: h<red> o<red> u<red> s<red> e<red> 
	  } 
	   } 
	   } 
	 & \\ 
 

    \theutterance \stepcounter{utterance}  

    & \multicolumn{4}{p{0.6\linewidth}}{\cellcolor[rgb]{0.8,1,0.9}{%
	\makecell[{{p{\linewidth}}}]{% 
	  \tt {\tiny [A$\rangle$GM]}  
	 <cot>\\ \tt The feedback indicates that none of the letters in "house" are in the word. This means I need to start fresh with a new set of letters. A good strategy could be to choose a word with a different set of common consonants and vowels, such as "bread" or "cloud", but another option is "fable" or "stamp" or "flute" but "fjord" is less common, so let's try "flute" then "stamp" then "fable" then "bread" in order, starting with "flute" then "stamp" if "flute" doesn't work, "fable" is the next guess.\\ \tt </cot>\\ \tt guess: flute 
	  } 
	   } 
	   } 
	 & & \\ 
 

    \theutterance \stepcounter{utterance}  

    & & & \multicolumn{2}{p{0.3\linewidth}}{\cellcolor[rgb]{0.95,0.95,0.95}{%
	\makecell[{{p{\linewidth}}}]{% 
	  \tt {\tiny [GM$|$GM]}  
	 attempts = 2\\ \tt target_word = party\\ \tt guess = flute\\ \tt guess_feedback = f<red> l<red> u<red> t<green> e<red> 
	  } 
	   } 
	   } 
	 & & \\ 
 

    \theutterance \stepcounter{utterance}  

    & & \multicolumn{4}{p{0.6\linewidth}}{\cellcolor[rgb]{0.9,0.9,0.9}{%
	\makecell[{{p{\linewidth}}}]{% 
	  \tt {\tiny [A$\langle$GM]}  
	 guess_feedback: f<red> l<red> u<red> t<green> e<red> 
	  } 
	   } 
	   } 
	 & \\ 
 

    \theutterance \stepcounter{utterance}  

    & \multicolumn{4}{p{0.6\linewidth}}{\cellcolor[rgb]{0.8,1,0.9}{%
	\makecell[{{p{\linewidth}}}]{% 
	  \tt {\tiny [A$\rangle$GM]}  
	 <cot>\\ \tt The feedback indicates that 't' is in the correct position, which is position 4, but 'f', 'l', 'u', and 'e' are not in the word. Since 't' is in position 4, my next guess should include 't' in the same position and introduce new letters to fill the other positions. A possible word could be "brunt" or "burnt" or "stint" but "brunt" has a 'u' which was previously marked red, so let's try "stint" or "burnt" then "brunt" in order, starting with "stint".\\ \tt </cot>\\ \tt guess: stint 
	  } 
	   } 
	   } 
	 & & \\ 
 

    \theutterance \stepcounter{utterance}  

    & & & \multicolumn{2}{p{0.3\linewidth}}{\cellcolor[rgb]{0.95,0.95,0.95}{%
	\makecell[{{p{\linewidth}}}]{% 
	  \tt {\tiny [GM$|$GM]}  
	 attempts = 3\\ \tt target_word = party\\ \tt guess = stint\\ \tt guess_feedback = s<red> t<yellow> i<red> n<red> t<red> 
	  } 
	   } 
	   } 
	 & & \\ 
 

    \theutterance \stepcounter{utterance}  

    & & \multicolumn{4}{p{0.6\linewidth}}{\cellcolor[rgb]{0.9,0.9,0.9}{%
	\makecell[{{p{\linewidth}}}]{% 
	  \tt {\tiny [A$\langle$GM]}  
	 guess_feedback: s<red> t<yellow> i<red> n<red> t<red> 
	  } 
	   } 
	   } 
	 & \\ 
 

    \theutterance \stepcounter{utterance}  

    & \multicolumn{4}{p{0.6\linewidth}}{\cellcolor[rgb]{0.8,1,0.9}{%
	\makecell[{{p{\linewidth}}}]{% 
	  \tt {\tiny [A$\rangle$GM]}  
	 <cot>\\ \tt The feedback indicates that 't' is in the word, but not in position 4 or 5, and 's', 'i', and 'n' are not in the word. Since 't' was previously green in position 4, but now it's yellow, it means 't' is in the word but not in position 4. Let me think of a new word that includes 't' in a different position and avoids the previously eliminated letters. A possible word could be "tryst" or "trots" but 's' was marked red, so let's try "trots" is not an option, how about "trust" then "tryst" in order, starting with "trust" then "tryst".\\ \tt </cot>\\ \tt guess: trust 
	  } 
	   } 
	   } 
	 & & \\ 
 

    \theutterance \stepcounter{utterance}  

    & & & \multicolumn{2}{p{0.3\linewidth}}{\cellcolor[rgb]{0.95,0.95,0.95}{%
	\makecell[{{p{\linewidth}}}]{% 
	  \tt {\tiny [GM$|$GM]}  
	 attempts = 4\\ \tt target_word = party\\ \tt guess = trust\\ \tt guess_feedback = t<yellow> r<yellow> u<red> s<red> t<red> 
	  } 
	   } 
	   } 
	 & & \\ 
 

    \theutterance \stepcounter{utterance}  

    & & \multicolumn{4}{p{0.6\linewidth}}{\cellcolor[rgb]{0.9,0.9,0.9}{%
	\makecell[{{p{\linewidth}}}]{% 
	  \tt {\tiny [A$\langle$GM]}  
	 guess_feedback: t<yellow> r<yellow> u<red> s<red> t<red> 
	  } 
	   } 
	   } 
	 & \\ 
 

    \theutterance \stepcounter{utterance}  

    & \multicolumn{4}{p{0.6\linewidth}}{\cellcolor[rgb]{0.8,1,0.9}{%
	\makecell[{{p{\linewidth}}}]{% 
	  \tt {\tiny [A$\rangle$GM]}  
	 <cot>\\ \tt The feedback indicates that 't' and 'r' are in the word, but not in the positions I previously tried. 'u' and 's' are not in the word. Since 't' was previously green in position 4, then yellow in a different position, and now it's yellow again, it means 't' is in the word but I need to keep trying different positions for it. Let me think of a new word that includes 't' and 'r' in different positions and avoids the previously eliminated letters. A possible word could be "trait" or "taper" or "rater" or "ratty" but 'u' was marked red and 'y' was not tried yet and 'a' and 'e' and 'y' could fit, so let's try "rater" or "taper" or "trait" in order, starting with "trait".\\ \tt </cot>\\ \tt guess: trait 
	  } 
	   } 
	   } 
	 & & \\ 
 

    \theutterance \stepcounter{utterance}  

    & & & \multicolumn{2}{p{0.3\linewidth}}{\cellcolor[rgb]{0.95,0.95,0.95}{%
	\makecell[{{p{\linewidth}}}]{% 
	  \tt {\tiny [GM$|$GM]}  
	 attempts = 5\\ \tt target_word = party\\ \tt guess = trait\\ \tt guess_feedback = t<yellow> r<yellow> a<yellow> i<red> t<red> 
	  } 
	   } 
	   } 
	 & & \\ 
 

    \theutterance \stepcounter{utterance}  

    & & \multicolumn{4}{p{0.6\linewidth}}{\cellcolor[rgb]{0.9,0.9,0.9}{%
	\makecell[{{p{\linewidth}}}]{% 
	  \tt {\tiny [A$\langle$GM]}  
	 guess_feedback: t<yellow> r<yellow> a<yellow> i<red> t<red> 
	  } 
	   } 
	   } 
	 & \\ 
 

    \theutterance \stepcounter{utterance}  

    & \multicolumn{4}{p{0.6\linewidth}}{\cellcolor[rgb]{0.8,1,0.9}{%
	\makecell[{{p{\linewidth}}}]{% 
	  \tt {\tiny [A$\rangle$GM]}  
	 <cot>\\ \tt The feedback indicates that 't', 'r', and 'a' are in the word, but not in the positions I previously tried. 'i' is not in the word. Since 't', 'r', and 'a' are all yellow, it means they are all in the word, but I need to find the correct positions for them. Let me think of a new word that includes 't', 'r', and 'a' in different positions and avoids the previously eliminated letters. A possible word could be "tarps" or "traps" or "ratios" but 'i', 's', 'o' were marked red or not tried yet and 'p' could fit, so let's try "traps" or "tarps" in order, starting with "traps".\\ \tt </cot>\\ \tt guess: traps 
	  } 
	   } 
	   } 
	 & & \\ 
 

    \theutterance \stepcounter{utterance}  

    & & & \multicolumn{2}{p{0.3\linewidth}}{\cellcolor[rgb]{0.95,0.95,0.95}{%
	\makecell[{{p{\linewidth}}}]{% 
	  \tt {\tiny [GM$|$GM]}  
	 attempts = 6\\ \tt target_word = party\\ \tt guess = traps\\ \tt guess_feedback = t<yellow> r<yellow> a<yellow> p<yellow> s<red> 
	  } 
	   } 
	   } 
	 & & \\ 
 

    \theutterance \stepcounter{utterance}  

    & & & \multicolumn{2}{p{0.3\linewidth}}{\cellcolor[rgb]{0.95,0.95,0.95}{%
	\makecell[{{p{\linewidth}}}]{% 
	  \tt {\tiny [GM$|$GM]}  
	 game_result = LOSS 
	  } 
	   } 
	   } 
	 & & \\ 
 

\end{supertabular}
}

\end{document}
