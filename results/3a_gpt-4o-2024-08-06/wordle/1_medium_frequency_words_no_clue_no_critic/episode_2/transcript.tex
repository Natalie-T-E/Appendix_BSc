
\documentclass{article}
\usepackage{colortbl}
\usepackage{makecell}
\usepackage{multirow}
\usepackage{supertabular}

\begin{document}

\newcounter{utterance}

\twocolumn

{ \footnotesize  \setcounter{utterance}{1}
\setlength{\tabcolsep}{0pt}
\begin{supertabular}{c@{$\;$}|p{.15\linewidth}@{}p{.15\linewidth}p{.15\linewidth}p{.15\linewidth}p{.15\linewidth}p{.15\linewidth}}

    \# & $\;$A & \multicolumn{4}{c}{Game Master} & $\;\:$B\\
    \hline 

    \theutterance \stepcounter{utterance}  

    & & \multicolumn{4}{p{0.6\linewidth}}{\cellcolor[rgb]{0.9,0.9,0.9}{%
	\makecell[{{p{\linewidth}}}]{% 
	  \tt {\tiny [A$\langle$GM]}  
	 You are a language wizard who likes to guess words by using the given rules.\\ \tt \\ \tt Welcome to Wordle! You have six attempts to guess the target word, a valid English word of five lowercase letters (a-z).\\ \tt After each guess, your answer will be validated, and you will receive feedback indicating which letters are correct (green), which letters are correct but in the wrong position (yellow), and which letters are incorrect (red). Utilize this feedback when determining which letters to include or exclude in your next guess.\\ \tt \\ \tt Please follow these instructions rigorously:\\ \tt \\ \tt In any case, the first token of your answer must be the tag "<cot>". The demanded reasoning steps to be enclosed in <cot></cot> depend on where you are in the game.\\ \tt \\ \tt For the starting guess:\\ \tt There is only one step within <cot></cot>: Provide a short explanation for your guess.\\ \tt Then state your guess, using the tag "guess:".\\ \tt \\ \tt For all subsequent guesses:\\ \tt Please adhere to the following 2 reasoning steps within <cot></cot>:\\ \tt 1) Explain the provided feedback. Be specific about both letters and positions (1-5). It can be helpful to mention relevant information from your earlier guess feedbacks as well. Be concise; you have a limited number of tokens!\\ \tt 2) Write a structured representation of what you know so far about the target word. Define a variable "keep" if you already encountered any green letters, and map these letters to their known positions. If there is only one remaining possible position for a yellow letter, it should also be included in the variable "keep". Define a variable "try_in_position" to formalize your conclusion regarding letters you know are in the target word but not yet where (when there is more than one position left to try), mapping these letters to their possible positions. Represent the set of eliminated letters by defining a variable "avoid". In addition, you may also define possible patterns. You are encouraged to use Python syntax in this step. You must finish this step with the closing tag </cot>.\\ \tt Then state your next guess — a valid 5-letter word in accordance with the identified constraints — using the tag "guess:".\\ \tt Important: If your guess consists of more or less than five letters, the game will be aborted!\\ \tt \\ \tt Do not generate any additional text.\\ \tt \\ \tt -----------\\ \tt Here is an example for a starting guess:\\ \tt Prompt: … Let's begin with your first guess.\\ \tt \\ \tt Example Answer:\\ \tt <cot>\\ \tt I'm starting with a common five-letter English word, which includes frequent consonants and vowels.\\ \tt </cot>\\ \tt guess: house\\ \tt \\ \tt -----------\\ \tt Here are two examples for a second/third/fourth/fifth/sixth guess:\\ \tt \\ \tt Example I:\\ \tt Context:\\ \tt 1st guess: "crane", guess_feedback: c<red> r<yellow> a<red> n<yellow> e<yellow>\\ \tt \\ \tt Answer:\\ \tt <cot>\\ \tt 1. Feedback explanation\\ \tt The letters 'r', 'n' and 'e' are in the word but not in positions 2, 4, 5, respectively. Also, 'c' and 'a' are not in the word (red).\\ \tt 2. Structured representation\\ \tt keep = {}  # no green letters yet\\ \tt try_in_position = {'r': {1, 3, 4, 5}, 'n': {1, 2, 3, 5}, 'e': {1, 2, 3, 4}}\\ \tt avoid = {'c', 'a'}\\ \tt </cot>\\ \tt guess: tenor\\ \tt \\ \tt Example II:\\ \tt Context:\\ \tt 1st guess: "lucky", guess_feedback: l<yellow> u<red> c<red> k<red> y<red>\\ \tt 2nd guess: "salad", guess_feedback: s<red> a<red> l<yellow> a<green> d<red>\\ \tt \\ \tt Answer:\\ \tt <cot>\\ \tt 1. Feedback explanation\\ \tt The feedback indicates that 'a' is correct in position 4 (and not in position 2). The letter 'l' is present but not in position 1 (yellow in "lucky") or 3 (yellow in "salad"). Moreover, position 4 is ruled out implicitly via the green 'a', this means 'l' must be in position 2 or 5. In addition to the previously eliminated letters {'u', 'c', 'k', 'y'}, 's' and 'd' are not in the word at all (red).\\ \tt 2. Structured representation\\ \tt keep = {'a': 4}\\ \tt try_in_position = {'l': {2, 5}}\\ \tt avoid = {'u', 'c', 'k', 'y', 's', 'd'}\\ \tt patterns = [\\ \tt     "_l_a_",  # 'l' in position 2\\ \tt     "???al",  # 'l' in position 5\\ \tt ]\\ \tt </cot>\\ \tt guess: trial\\ \tt \\ \tt -----------\\ \tt Let's begin with your first guess. 
	  } 
	   } 
	   } 
	 & \\ 
 

    \theutterance \stepcounter{utterance}  

    & \multicolumn{4}{p{0.6\linewidth}}{\cellcolor[rgb]{0.8,1,0.9}{%
	\makecell[{{p{\linewidth}}}]{% 
	  \tt {\tiny [A$\rangle$GM]}  
	 <cot>\\ \tt I'm starting with a common five-letter English word, which includes frequent consonants and vowels.\\ \tt </cot>\\ \tt guess: slate 
	  } 
	   } 
	   } 
	 & & \\ 
 

    \theutterance \stepcounter{utterance}  

    & & & \multicolumn{2}{p{0.3\linewidth}}{\cellcolor[rgb]{0.95,0.95,0.95}{%
	\makecell[{{p{\linewidth}}}]{% 
	  \tt {\tiny [GM$|$GM]}  
	 attempts = 1\\ \tt target_word = loyal\\ \tt guess = slate\\ \tt guess_feedback = s<red> l<yellow> a<yellow> t<red> e<red> 
	  } 
	   } 
	   } 
	 & & \\ 
 

    \theutterance \stepcounter{utterance}  

    & & \multicolumn{4}{p{0.6\linewidth}}{\cellcolor[rgb]{0.9,0.9,0.9}{%
	\makecell[{{p{\linewidth}}}]{% 
	  \tt {\tiny [A$\langle$GM]}  
	 guess_feedback: s<red> l<yellow> a<yellow> t<red> e<red> 
	  } 
	   } 
	   } 
	 & \\ 
 

    \theutterance \stepcounter{utterance}  

    & \multicolumn{4}{p{0.6\linewidth}}{\cellcolor[rgb]{0.8,1,0.9}{%
	\makecell[{{p{\linewidth}}}]{% 
	  \tt {\tiny [A$\rangle$GM]}  
	 <cot>\\ \tt 1. Feedback explanation\\ \tt The letters 'l' and 'a' are in the word but not in positions 2 and 3, respectively. The letters 's', 't', and 'e' are not in the word (red).\\ \tt 2. Structured representation\\ \tt keep = {}  # no green letters yet\\ \tt try_in_position = {'l': {1, 3, 4, 5}, 'a': {1, 2, 4, 5}}\\ \tt avoid = {'s', 't', 'e'}\\ \tt </cot>\\ \tt guess: coral 
	  } 
	   } 
	   } 
	 & & \\ 
 

    \theutterance \stepcounter{utterance}  

    & & & \multicolumn{2}{p{0.3\linewidth}}{\cellcolor[rgb]{0.95,0.95,0.95}{%
	\makecell[{{p{\linewidth}}}]{% 
	  \tt {\tiny [GM$|$GM]}  
	 attempts = 2\\ \tt target_word = loyal\\ \tt guess = coral\\ \tt guess_feedback = c<red> o<green> r<red> a<green> l<green> 
	  } 
	   } 
	   } 
	 & & \\ 
 

    \theutterance \stepcounter{utterance}  

    & & \multicolumn{4}{p{0.6\linewidth}}{\cellcolor[rgb]{0.9,0.9,0.9}{%
	\makecell[{{p{\linewidth}}}]{% 
	  \tt {\tiny [A$\langle$GM]}  
	 guess_feedback: c<red> o<green> r<red> a<green> l<green> 
	  } 
	   } 
	   } 
	 & \\ 
 

    \theutterance \stepcounter{utterance}  

    & \multicolumn{4}{p{0.6\linewidth}}{\cellcolor[rgb]{0.8,1,0.9}{%
	\makecell[{{p{\linewidth}}}]{% 
	  \tt {\tiny [A$\rangle$GM]}  
	 <cot>\\ \tt 1. Feedback explanation\\ \tt The feedback indicates that 'o' is correct in position 2, 'a' is correct in position 4, and 'l' is correct in position 5. The letters 'c' and 'r' are not in the word (red).\\ \tt 2. Structured representation\\ \tt keep = {'o': 2, 'a': 4, 'l': 5}\\ \tt try_in_position = {}  # all known letters are in their correct positions\\ \tt avoid = {'s', 't', 'e', 'c', 'r'}\\ \tt </cot>\\ \tt guess: opal 
	  } 
	   } 
	   } 
	 & & \\ 
 

    \theutterance \stepcounter{utterance}  

    & & & \multicolumn{2}{p{0.3\linewidth}}{\cellcolor[rgb]{0.95,0.95,0.95}{%
	\makecell[{{p{\linewidth}}}]{% 
	  \tt {\tiny [GM$|$GM]}  
	 Guess does not conform to the format rules\\ \tt Error: The length of the guessed word is not 5. 
	  } 
	   } 
	   } 
	 & & \\ 
 

    \theutterance \stepcounter{utterance}  

    & & & \multicolumn{2}{p{0.3\linewidth}}{\cellcolor[rgb]{0.95,0.95,0.95}{%
	\makecell[{{p{\linewidth}}}]{% 
	  \tt {\tiny [GM$|$GM]}  
	 game_result = ABORT 
	  } 
	   } 
	   } 
	 & & \\ 
 

\end{supertabular}
}

\end{document}
