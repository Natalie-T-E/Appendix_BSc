
\documentclass{article}
\usepackage{colortbl}
\usepackage{makecell}
\usepackage{multirow}
\usepackage{supertabular}

\begin{document}

\newcounter{utterance}

\twocolumn

{ \footnotesize  \setcounter{utterance}{1}
\setlength{\tabcolsep}{0pt}
\begin{supertabular}{c@{$\;$}|p{.15\linewidth}@{}p{.15\linewidth}p{.15\linewidth}p{.15\linewidth}p{.15\linewidth}p{.15\linewidth}}

    \# & $\;$A & \multicolumn{4}{c}{Game Master} & $\;\:$B\\
    \hline 

    \theutterance \stepcounter{utterance}  

    & & \multicolumn{4}{p{0.6\linewidth}}{\cellcolor[rgb]{0.9,0.9,0.9}{%
	\makecell[{{p{\linewidth}}}]{% 
	  \tt {\tiny [A$\langle$GM]}  
	 You are a language wizard who likes to guess words by using the given rules.\\ \tt \\ \tt Welcome to Wordle! You have six attempts to guess the target word, a valid English word of five lowercase letters (a-z).\\ \tt After each guess, your answer will be validated, and you will receive feedback indicating which letters are correct (green), which letters are correct but in the wrong position (yellow), and which letters are incorrect (red). Utilize this feedback when determining which letters to include or exclude in your next guess.\\ \tt \\ \tt For example, the feedback for "storm" might be:\\ \tt guess_feedback: s<yellow> t<yellow> o<green> r<yellow> m<red>\\ \tt \\ \tt Please follow these instructions rigorously:\\ \tt \\ \tt In any case, the first token of your answer must be the tag "<cot>". Then explain how you are using the guess_feedback to arrive at a new guess. Let's think step by step. Your reasoning steps may depend on where you are in the game.\\ \tt Be concise; you have a limited number of tokens!\\ \tt After finishing your reasoning with the closing tag </cot>: State your guess, using the tag "guess:".\\ \tt Do not generate any additional text.\\ \tt \\ \tt For instance, if your guess is "apple", your response must be formatted as follows:\\ \tt <cot>\\ \tt your step-by-step reasoning\\ \tt </cot>\\ \tt guess: apple\\ \tt \\ \tt Important: If your guess consists of more or less than five letters, the game will be aborted!\\ \tt \\ \tt Let's begin with your first guess. 
	  } 
	   } 
	   } 
	 & \\ 
 

    \theutterance \stepcounter{utterance}  

    & \multicolumn{4}{p{0.6\linewidth}}{\cellcolor[rgb]{0.8,1,0.9}{%
	\makecell[{{p{\linewidth}}}]{% 
	  \tt {\tiny [A$\rangle$GM]}  
	 <cot>\\ \tt To start, I'll choose a common five-letter word with a good balance of vowels and consonants. A word like "house" is a reasonable initial guess because it contains a mix of letters that are frequently used in many English words.\\ \tt </cot>\\ \tt guess: house 
	  } 
	   } 
	   } 
	 & & \\ 
 

    \theutterance \stepcounter{utterance}  

    & & & \multicolumn{2}{p{0.3\linewidth}}{\cellcolor[rgb]{0.95,0.95,0.95}{%
	\makecell[{{p{\linewidth}}}]{% 
	  \tt {\tiny [GM$|$GM]}  
	 attempts = 1\\ \tt target_word = linen\\ \tt guess = house\\ \tt guess_feedback = h<red> o<red> u<red> s<red> e<yellow> 
	  } 
	   } 
	   } 
	 & & \\ 
 

    \theutterance \stepcounter{utterance}  

    & & \multicolumn{4}{p{0.6\linewidth}}{\cellcolor[rgb]{0.9,0.9,0.9}{%
	\makecell[{{p{\linewidth}}}]{% 
	  \tt {\tiny [A$\langle$GM]}  
	 guess_feedback: h<red> o<red> u<red> s<red> e<yellow> 
	  } 
	   } 
	   } 
	 & \\ 
 

    \theutterance \stepcounter{utterance}  

    & \multicolumn{4}{p{0.6\linewidth}}{\cellcolor[rgb]{0.8,1,0.9}{%
	\makecell[{{p{\linewidth}}}]{% 
	  \tt {\tiny [A$\rangle$GM]}  
	 <cot>\\ \tt The feedback indicates that "h", "o", "u", and "s" are incorrect, while "e" is correct but in the wrong position. For my next guess, I'll keep "e" and introduce new letters that are commonly used in English words, trying to utilize "e" in a different position.\\ \tt </cot>\\ \tt guess: fence 
	  } 
	   } 
	   } 
	 & & \\ 
 

    \theutterance \stepcounter{utterance}  

    & & & \multicolumn{2}{p{0.3\linewidth}}{\cellcolor[rgb]{0.95,0.95,0.95}{%
	\makecell[{{p{\linewidth}}}]{% 
	  \tt {\tiny [GM$|$GM]}  
	 attempts = 2\\ \tt target_word = linen\\ \tt guess = fence\\ \tt guess_feedback = f<red> e<yellow> n<green> c<red> e<red> 
	  } 
	   } 
	   } 
	 & & \\ 
 

    \theutterance \stepcounter{utterance}  

    & & \multicolumn{4}{p{0.6\linewidth}}{\cellcolor[rgb]{0.9,0.9,0.9}{%
	\makecell[{{p{\linewidth}}}]{% 
	  \tt {\tiny [A$\langle$GM]}  
	 guess_feedback: f<red> e<yellow> n<green> c<red> e<red> 
	  } 
	   } 
	   } 
	 & \\ 
 

    \theutterance \stepcounter{utterance}  

    & \multicolumn{4}{p{0.6\linewidth}}{\cellcolor[rgb]{0.8,1,0.9}{%
	\makecell[{{p{\linewidth}}}]{% 
	  \tt {\tiny [A$\rangle$GM]}  
	 <cot>\\ \tt The feedback shows that "f" and "c" are incorrect, "e" is still correct but in the wrong position, and "n" is correct and in the right position. I'll keep "n" in the same position and "e" in the guess, trying a different position for "e" again, and introduce new letters.\\ \tt </cot>\\ \tt guess: nerve 
	  } 
	   } 
	   } 
	 & & \\ 
 

    \theutterance \stepcounter{utterance}  

    & & & \multicolumn{2}{p{0.3\linewidth}}{\cellcolor[rgb]{0.95,0.95,0.95}{%
	\makecell[{{p{\linewidth}}}]{% 
	  \tt {\tiny [GM$|$GM]}  
	 attempts = 3\\ \tt target_word = linen\\ \tt guess = nerve\\ \tt guess_feedback = n<yellow> e<yellow> r<red> v<red> e<red> 
	  } 
	   } 
	   } 
	 & & \\ 
 

    \theutterance \stepcounter{utterance}  

    & & \multicolumn{4}{p{0.6\linewidth}}{\cellcolor[rgb]{0.9,0.9,0.9}{%
	\makecell[{{p{\linewidth}}}]{% 
	  \tt {\tiny [A$\langle$GM]}  
	 guess_feedback: n<yellow> e<yellow> r<red> v<red> e<red> 
	  } 
	   } 
	   } 
	 & \\ 
 

    \theutterance \stepcounter{utterance}  

    & \multicolumn{4}{p{0.6\linewidth}}{\cellcolor[rgb]{0.8,1,0.9}{%
	\makecell[{{p{\linewidth}}}]{% 
	  \tt {\tiny [A$\rangle$GM]}  
	 <cot>\\ \tt The feedback indicates that "n" and "e" are correct but in the wrong positions, and "r" and "v" are incorrect. I'll try to utilize "n" and "e" in different positions, introducing new letters that are commonly paired with "n" and "e".\\ \tt </cot>\\ \tt guess: neigh 
	  } 
	   } 
	   } 
	 & & \\ 
 

    \theutterance \stepcounter{utterance}  

    & & & \multicolumn{2}{p{0.3\linewidth}}{\cellcolor[rgb]{0.95,0.95,0.95}{%
	\makecell[{{p{\linewidth}}}]{% 
	  \tt {\tiny [GM$|$GM]}  
	 attempts = 4\\ \tt target_word = linen\\ \tt guess = neigh\\ \tt guess_feedback = n<yellow> e<yellow> i<yellow> g<red> h<red> 
	  } 
	   } 
	   } 
	 & & \\ 
 

    \theutterance \stepcounter{utterance}  

    & & \multicolumn{4}{p{0.6\linewidth}}{\cellcolor[rgb]{0.9,0.9,0.9}{%
	\makecell[{{p{\linewidth}}}]{% 
	  \tt {\tiny [A$\langle$GM]}  
	 guess_feedback: n<yellow> e<yellow> i<yellow> g<red> h<red> 
	  } 
	   } 
	   } 
	 & \\ 
 

    \theutterance \stepcounter{utterance}  

    & \multicolumn{4}{p{0.6\linewidth}}{\cellcolor[rgb]{0.8,1,0.9}{%
	\makecell[{{p{\linewidth}}}]{% 
	  \tt {\tiny [A$\rangle$GM]}  
	 <cot>\\ \tt The feedback shows that "n", "e", and "i" are correct but in the wrong positions, and "g" and "h" are incorrect. I'll keep "n", "e", and "i" in the guess, trying different positions, and introduce new letters to complete the word.\\ \tt </cot>\\ \tt guess: niece 
	  } 
	   } 
	   } 
	 & & \\ 
 

    \theutterance \stepcounter{utterance}  

    & & & \multicolumn{2}{p{0.3\linewidth}}{\cellcolor[rgb]{0.95,0.95,0.95}{%
	\makecell[{{p{\linewidth}}}]{% 
	  \tt {\tiny [GM$|$GM]}  
	 attempts = 5\\ \tt target_word = linen\\ \tt guess = niece\\ \tt guess_feedback = n<yellow> i<green> e<yellow> c<red> e<red> 
	  } 
	   } 
	   } 
	 & & \\ 
 

    \theutterance \stepcounter{utterance}  

    & & \multicolumn{4}{p{0.6\linewidth}}{\cellcolor[rgb]{0.9,0.9,0.9}{%
	\makecell[{{p{\linewidth}}}]{% 
	  \tt {\tiny [A$\langle$GM]}  
	 guess_feedback: n<yellow> i<green> e<yellow> c<red> e<red> 
	  } 
	   } 
	   } 
	 & \\ 
 

    \theutterance \stepcounter{utterance}  

    & \multicolumn{4}{p{0.6\linewidth}}{\cellcolor[rgb]{0.8,1,0.9}{%
	\makecell[{{p{\linewidth}}}]{% 
	  \tt {\tiny [A$\rangle$GM]}  
	 <cot>\\ \tt The feedback indicates that "i" is correct and in the right position, "n" and "e" are correct but in the wrong positions, and "c" is incorrect. I'll keep "i" in the same position, try to utilize "n" and "e" in different positions, and introduce new letters.\\ \tt </cot>\\ \tt guess: field 
	  } 
	   } 
	   } 
	 & & \\ 
 

    \theutterance \stepcounter{utterance}  

    & & & \multicolumn{2}{p{0.3\linewidth}}{\cellcolor[rgb]{0.95,0.95,0.95}{%
	\makecell[{{p{\linewidth}}}]{% 
	  \tt {\tiny [GM$|$GM]}  
	 attempts = 6\\ \tt target_word = linen\\ \tt guess = field\\ \tt guess_feedback = f<red> i<green> e<yellow> l<yellow> d<red> 
	  } 
	   } 
	   } 
	 & & \\ 
 

    \theutterance \stepcounter{utterance}  

    & & & \multicolumn{2}{p{0.3\linewidth}}{\cellcolor[rgb]{0.95,0.95,0.95}{%
	\makecell[{{p{\linewidth}}}]{% 
	  \tt {\tiny [GM$|$GM]}  
	 game_result = LOSS 
	  } 
	   } 
	   } 
	 & & \\ 
 

\end{supertabular}
}

\end{document}
