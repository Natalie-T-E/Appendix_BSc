
\documentclass{article}
\usepackage{colortbl}
\usepackage{makecell}
\usepackage{multirow}
\usepackage{supertabular}

\begin{document}

\newcounter{utterance}

\twocolumn

{ \footnotesize  \setcounter{utterance}{1}
\setlength{\tabcolsep}{0pt}
\begin{supertabular}{c@{$\;$}|p{.15\linewidth}@{}p{.15\linewidth}p{.15\linewidth}p{.15\linewidth}p{.15\linewidth}p{.15\linewidth}}

    \# & $\;$A & \multicolumn{4}{c}{Game Master} & $\;\:$B\\
    \hline 

    \theutterance \stepcounter{utterance}  

    & & \multicolumn{4}{p{0.6\linewidth}}{\cellcolor[rgb]{0.9,0.9,0.9}{%
	\makecell[{{p{\linewidth}}}]{% 
	  \tt {\tiny [A$\langle$GM]}  
	 You are a language wizard who likes to guess words by using the given rules.\\ \tt \\ \tt Welcome to Wordle! You have six attempts to guess the target word, a valid English word of five lowercase letters (a-z).\\ \tt After each guess, your answer will be validated, and you will receive feedback indicating which letters are correct (green), which letters are correct but in the wrong position (yellow), and which letters are incorrect (red). Utilize this feedback when determining which letters to include or exclude in your next guess.\\ \tt \\ \tt For example, the feedback for "storm" might be:\\ \tt guess_feedback: s<yellow> t<yellow> o<green> r<yellow> m<red>\\ \tt \\ \tt Please follow these instructions rigorously:\\ \tt \\ \tt In any case, the first token of your answer must be the tag "<cot>". Then explain how you are using the guess_feedback to arrive at a new guess. Let's think step by step. Your reasoning steps may depend on where you are in the game.\\ \tt Be concise; you have a limited number of tokens!\\ \tt After finishing your reasoning with the closing tag </cot>: State your guess, using the tag "guess:".\\ \tt Do not generate any additional text.\\ \tt \\ \tt For instance, if your guess is "apple", your response must be formatted as follows:\\ \tt <cot>\\ \tt your step-by-step reasoning\\ \tt </cot>\\ \tt guess: apple\\ \tt \\ \tt Important: If your guess consists of more or less than five letters, the game will be aborted!\\ \tt \\ \tt Let's begin with your first guess. 
	  } 
	   } 
	   } 
	 & \\ 
 

    \theutterance \stepcounter{utterance}  

    & \multicolumn{4}{p{0.6\linewidth}}{\cellcolor[rgb]{0.8,1,0.9}{%
	\makecell[{{p{\linewidth}}}]{% 
	  \tt {\tiny [A$\rangle$GM]}  
	 <cot>\\ \tt To start, I'll choose a common five-letter word with a good balance of vowels and consonants. A word like "house" is a reasonable initial guess because it contains a mix of letters that are frequently used in many English words.\\ \tt </cot>\\ \tt guess: house 
	  } 
	   } 
	   } 
	 & & \\ 
 

    \theutterance \stepcounter{utterance}  

    & & & \multicolumn{2}{p{0.3\linewidth}}{\cellcolor[rgb]{0.95,0.95,0.95}{%
	\makecell[{{p{\linewidth}}}]{% 
	  \tt {\tiny [GM$|$GM]}  
	 attempts = 1\\ \tt target_word = never\\ \tt guess = house\\ \tt guess_feedback = h<red> o<red> u<red> s<red> e<yellow> 
	  } 
	   } 
	   } 
	 & & \\ 
 

    \theutterance \stepcounter{utterance}  

    & & \multicolumn{4}{p{0.6\linewidth}}{\cellcolor[rgb]{0.9,0.9,0.9}{%
	\makecell[{{p{\linewidth}}}]{% 
	  \tt {\tiny [A$\langle$GM]}  
	 guess_feedback: h<red> o<red> u<red> s<red> e<yellow> 
	  } 
	   } 
	   } 
	 & \\ 
 

    \theutterance \stepcounter{utterance}  

    & \multicolumn{4}{p{0.6\linewidth}}{\cellcolor[rgb]{0.8,1,0.9}{%
	\makecell[{{p{\linewidth}}}]{% 
	  \tt {\tiny [A$\rangle$GM]}  
	 <cot>\\ \tt The feedback indicates that "h", "o", "u", and "s" are incorrect, while "e" is correct but in the wrong position. For my next guess, I'll keep "e" and introduce new letters that are commonly used in English words, trying to utilize "e" in a different position. A word like "teach" or "fence" could be a good option, but I'll choose "fence" as it has a good mix of vowels and consonants.\\ \tt </cot>\\ \tt guess: fence 
	  } 
	   } 
	   } 
	 & & \\ 
 

    \theutterance \stepcounter{utterance}  

    & & & \multicolumn{2}{p{0.3\linewidth}}{\cellcolor[rgb]{0.95,0.95,0.95}{%
	\makecell[{{p{\linewidth}}}]{% 
	  \tt {\tiny [GM$|$GM]}  
	 attempts = 2\\ \tt target_word = never\\ \tt guess = fence\\ \tt guess_feedback = f<red> e<green> n<yellow> c<red> e<yellow> 
	  } 
	   } 
	   } 
	 & & \\ 
 

    \theutterance \stepcounter{utterance}  

    & & \multicolumn{4}{p{0.6\linewidth}}{\cellcolor[rgb]{0.9,0.9,0.9}{%
	\makecell[{{p{\linewidth}}}]{% 
	  \tt {\tiny [A$\langle$GM]}  
	 guess_feedback: f<red> e<green> n<yellow> c<red> e<yellow> 
	  } 
	   } 
	   } 
	 & \\ 
 

    \theutterance \stepcounter{utterance}  

    & \multicolumn{4}{p{0.6\linewidth}}{\cellcolor[rgb]{0.8,1,0.9}{%
	\makecell[{{p{\linewidth}}}]{% 
	  \tt {\tiny [A$\rangle$GM]}  
	 <cot>\\ \tt The feedback shows that "e" is now in the correct position, "n" and the second "e" are correct but in the wrong positions, while "f" and "c" are incorrect. For my next guess, I'll keep the correctly positioned "e" and try to utilize "n" and the second "e" in different positions, introducing new letters to replace "f" and "c". A word like "needy" could be a good fit.\\ \tt </cot>\\ \tt guess: needy 
	  } 
	   } 
	   } 
	 & & \\ 
 

    \theutterance \stepcounter{utterance}  

    & & & \multicolumn{2}{p{0.3\linewidth}}{\cellcolor[rgb]{0.95,0.95,0.95}{%
	\makecell[{{p{\linewidth}}}]{% 
	  \tt {\tiny [GM$|$GM]}  
	 attempts = 3\\ \tt target_word = never\\ \tt guess = needy\\ \tt guess_feedback = n<green> e<green> e<yellow> d<red> y<red> 
	  } 
	   } 
	   } 
	 & & \\ 
 

    \theutterance \stepcounter{utterance}  

    & & \multicolumn{4}{p{0.6\linewidth}}{\cellcolor[rgb]{0.9,0.9,0.9}{%
	\makecell[{{p{\linewidth}}}]{% 
	  \tt {\tiny [A$\langle$GM]}  
	 guess_feedback: n<green> e<green> e<yellow> d<red> y<red> 
	  } 
	   } 
	   } 
	 & \\ 
 

    \theutterance \stepcounter{utterance}  

    & \multicolumn{4}{p{0.6\linewidth}}{\cellcolor[rgb]{0.8,1,0.9}{%
	\makecell[{{p{\linewidth}}}]{% 
	  \tt {\tiny [A$\rangle$GM]}  
	 <cot>\\ \tt The feedback indicates that "n" and the first "e" are now in the correct positions, the second "e" is correct but in the wrong position, and "d" and "y" are incorrect. For my next guess, I'll keep the correctly positioned "n" and "e", and try to utilize the second "e" in a different position, introducing new letters to replace "d" and "y". A word like "needy" was close, so I'll try a similar word, "needy" without the "d" and "y", which leads me to think of "neeve" or "neese" but a more common word is "neese" is not valid, a valid word is "geese" or "kneel" or "neese" is not valid, a valid word is "geese" or "sneer" or "snee" is not valid, a valid word is "geese" or "sneer".\\ \tt </cot>\\ \tt guess: sneer 
	  } 
	   } 
	   } 
	 & & \\ 
 

    \theutterance \stepcounter{utterance}  

    & & & \multicolumn{2}{p{0.3\linewidth}}{\cellcolor[rgb]{0.95,0.95,0.95}{%
	\makecell[{{p{\linewidth}}}]{% 
	  \tt {\tiny [GM$|$GM]}  
	 attempts = 4\\ \tt target_word = never\\ \tt guess = sneer\\ \tt guess_feedback = s<red> n<yellow> e<yellow> e<green> r<green> 
	  } 
	   } 
	   } 
	 & & \\ 
 

    \theutterance \stepcounter{utterance}  

    & & \multicolumn{4}{p{0.6\linewidth}}{\cellcolor[rgb]{0.9,0.9,0.9}{%
	\makecell[{{p{\linewidth}}}]{% 
	  \tt {\tiny [A$\langle$GM]}  
	 guess_feedback: s<red> n<yellow> e<yellow> e<green> r<green> 
	  } 
	   } 
	   } 
	 & \\ 
 

    \theutterance \stepcounter{utterance}  

    & \multicolumn{4}{p{0.6\linewidth}}{\cellcolor[rgb]{0.8,1,0.9}{%
	\makecell[{{p{\linewidth}}}]{% 
	  \tt {\tiny [A$\rangle$GM]}  
	 <cot>\\ \tt The feedback shows that "s" is incorrect, "n" and the first "e" are correct but in the wrong positions, and the second "e" and "r" are now in the correct positions. For my next guess, I'll keep the correctly positioned "e" and "r", and try to utilize "n" and the first "e" in different positions, introducing new letters to replace "s". A word like "nerve" could be a good fit.\\ \tt </cot>\\ \tt guess: nerve 
	  } 
	   } 
	   } 
	 & & \\ 
 

    \theutterance \stepcounter{utterance}  

    & & & \multicolumn{2}{p{0.3\linewidth}}{\cellcolor[rgb]{0.95,0.95,0.95}{%
	\makecell[{{p{\linewidth}}}]{% 
	  \tt {\tiny [GM$|$GM]}  
	 attempts = 5\\ \tt target_word = never\\ \tt guess = nerve\\ \tt guess_feedback = n<green> e<green> r<yellow> v<yellow> e<yellow> 
	  } 
	   } 
	   } 
	 & & \\ 
 

    \theutterance \stepcounter{utterance}  

    & & \multicolumn{4}{p{0.6\linewidth}}{\cellcolor[rgb]{0.9,0.9,0.9}{%
	\makecell[{{p{\linewidth}}}]{% 
	  \tt {\tiny [A$\langle$GM]}  
	 guess_feedback: n<green> e<green> r<yellow> v<yellow> e<yellow> 
	  } 
	   } 
	   } 
	 & \\ 
 

    \theutterance \stepcounter{utterance}  

    & \multicolumn{4}{p{0.6\linewidth}}{\cellcolor[rgb]{0.8,1,0.9}{%
	\makecell[{{p{\linewidth}}}]{% 
	  \tt {\tiny [A$\rangle$GM]}  
	 <cot>\\ \tt The feedback indicates that "n" and the first "e" are now in the correct positions, and "r", "v", and the second "e" are correct but in the wrong positions. For my next guess, I'll keep the correctly positioned "n" and "e", and try to utilize "r", "v", and the second "e" in different positions. A word like "never" could be a good fit.\\ \tt </cot>\\ \tt guess: never 
	  } 
	   } 
	   } 
	 & & \\ 
 

    \theutterance \stepcounter{utterance}  

    & & & \multicolumn{2}{p{0.3\linewidth}}{\cellcolor[rgb]{0.95,0.95,0.95}{%
	\makecell[{{p{\linewidth}}}]{% 
	  \tt {\tiny [GM$|$GM]}  
	 attempts = 6\\ \tt target_word = never\\ \tt guess = never\\ \tt guess_feedback = n<green> e<green> v<green> e<green> r<green> 
	  } 
	   } 
	   } 
	 & & \\ 
 

    \theutterance \stepcounter{utterance}  

    & & & \multicolumn{2}{p{0.3\linewidth}}{\cellcolor[rgb]{0.95,0.95,0.95}{%
	\makecell[{{p{\linewidth}}}]{% 
	  \tt {\tiny [GM$|$GM]}  
	 game_result = WIN 
	  } 
	   } 
	   } 
	 & & \\ 
 

\end{supertabular}
}

\end{document}
