
\documentclass{article}
\usepackage{colortbl}
\usepackage{makecell}
\usepackage{multirow}
\usepackage{supertabular}

\begin{document}

\newcounter{utterance}

\twocolumn

{ \footnotesize  \setcounter{utterance}{1}
\setlength{\tabcolsep}{0pt}
\begin{supertabular}{c@{$\;$}|p{.15\linewidth}@{}p{.15\linewidth}p{.15\linewidth}p{.15\linewidth}p{.15\linewidth}p{.15\linewidth}}

    \# & $\;$A & \multicolumn{4}{c}{Game Master} & $\;\:$B\\
    \hline 

    \theutterance \stepcounter{utterance}  

    & & \multicolumn{4}{p{0.6\linewidth}}{\cellcolor[rgb]{0.9,0.9,0.9}{%
	\makecell[{{p{\linewidth}}}]{% 
	  \tt {\tiny [A$\langle$GM]}  
	 You are a language wizard who likes to guess words by using the given rules.\\ \tt \\ \tt Welcome to Wordle! You have six attempts to guess the target word, a valid English word of five lowercase letters (a-z).\\ \tt After each guess, your answer will be validated, and you will receive feedback indicating which letters are correct (green), which letters are correct but in the wrong position (yellow), and which letters are incorrect (red). Utilize this feedback when determining which letters to include or exclude in your next guess.\\ \tt \\ \tt Please follow these instructions rigorously:\\ \tt \\ \tt In any case, the first token of your answer must be the tag "<cot>". The demanded reasoning steps to be enclosed in <cot></cot> depend on where you are in the game.\\ \tt \\ \tt For the starting guess:\\ \tt There is only one step within <cot></cot>: Provide a short explanation for your guess.\\ \tt Then state your guess, using the tag "guess:".\\ \tt \\ \tt For all subsequent guesses:\\ \tt Please adhere to the following 3 reasoning steps within <cot></cot>:\\ \tt 1) Explain the provided feedback. Be specific about both letters and positions (1-5). It can be helpful to mention relevant information from your earlier guess feedbacks as well.\\ \tt 2) Write a structured representation of what you know so far about the target word. Define a variable "keep" if you already encountered any green letters, and map these letters to their known positions. If there is only one remaining possible position for a yellow letter, it should also be included in the variable "keep". Define a variable "try_in_position" to formalize your conclusion regarding letters you know are in the target word but not yet where (when there is more than one position left to try), mapping these letters to their possible positions. Represent the set of eliminated letters by defining a variable "avoid". In addition, you may also define possible patterns. You are encouraged to use Python syntax in this step.\\ \tt 3) Find your next guess: a valid 5-letter word in accordance with the identified constraints. Important: Take your first idea that does not contradict any constraints. Do not forget the word length constraint. If your guess consists of more or less than five letters, the game will be aborted! Be concise; you have a limited number of tokens! You must finish this step with the closing tag </cot>.\\ \tt Then state the guess you chose in step 3 once more, using the tag "guess:". Please do NOT come up with a new guess at this point.\\ \tt \\ \tt Do not generate any additional text.\\ \tt \\ \tt -----------\\ \tt Here is an example for a starting guess:\\ \tt Prompt: … Let's begin with your first guess.\\ \tt \\ \tt Example Answer:\\ \tt <cot>\\ \tt I'm starting with a common five-letter English word, which includes frequent consonants and vowels.\\ \tt </cot>\\ \tt guess: house\\ \tt \\ \tt -----------\\ \tt Here are two examples for a second/third/fourth/fifth/sixth guess:\\ \tt \\ \tt Example I:\\ \tt Context:\\ \tt 1st guess: "crane", guess_feedback: c<red> r<yellow> a<red> n<yellow> e<yellow>\\ \tt \\ \tt Answer:\\ \tt <cot>\\ \tt 1. Feedback explanation\\ \tt The letters 'r', 'n' and 'e' are in the word but not in positions 2, 4, 5, respectively. Also, 'c' and 'a' are not in the word (red).\\ \tt 2. Structured representation\\ \tt keep = {}  # no green letters yet\\ \tt try_in_position = {'r': {1, 3, 4, 5}, 'n': {1, 2, 3, 5}, 'e': {1, 2, 3, 4}}\\ \tt avoid = {'c', 'a'}\\ \tt 3. A fitting 5-letter word\\ \tt Let me think of a valid 5-letter word that fits all our constraints: So I need to avoid {'c', 'a'} and introduce two new letters instead (for example, 's', 't', 'l', 'f', 'o', 'u', 'i' or 'y'). Also, I'm trying to find a new arrangement for the yellow letters 'r', 'n' and 'e'. Maybe try them in reverse order? "en" could actually be a substring, possibly followed by a new letter and then by 'r' — as in "_en_r" (alternatively, "en_r_"). "tenor" comes to mind. Let me check the constraints: 'e' is now placed in position 2, which is indeed in try_in_position['e'] (fine), 'n' is moved to position 3, which is in try_in_position['n'] (fine), 'r' is now in position 5, which is in try_in_position['r'] (fine). "tenor" doesn't include any eliminated letters. All criteria are fulfilled, so "tenor" is a good next guess.\\ \tt </cot>\\ \tt guess: tenor\\ \tt \\ \tt Example II:\\ \tt Context:\\ \tt 1st guess: "lucky", guess_feedback: l<yellow> u<red> c<red> k<red> y<red>\\ \tt 2nd guess: "salad", guess_feedback: s<red> a<red> l<yellow> a<green> d<red>\\ \tt \\ \tt Answer:\\ \tt <cot>\\ \tt 1. Feedback explanation\\ \tt The feedback indicates that 'a' is correct in position 4 (and not in position 2). The letter 'l' is present but not in position 1 (yellow in "lucky") or 3 (yellow in "salad"). Moreover, position 4 is ruled out implicitly via the green 'a', this means 'l' must be in position 2 or 5. In addition to the previously eliminated letters {'u', 'c', 'k', 'y'}, 's' and 'd' are not in the word at all (red).\\ \tt 2. Structured representation\\ \tt keep = {'a': 4}\\ \tt try_in_position = {'l': {2, 5}}\\ \tt avoid = {'u', 'c', 'k', 'y', 's', 'd'}\\ \tt patterns = [\\ \tt     "_l_a_",  # 'l' in position 2\\ \tt     "???al",  # 'l' in position 5\\ \tt ]\\ \tt 3. A fitting 5-letter word\\ \tt "???al" (patterns[1]) seems like a good starting point. To fill in the placeholders, I need three new letters. Common consonants I haven't tried yet include: 't', 'r', 'n'. How about "trial" (T-R-I-A-L):\\ \tt — t in position 1 (new letter)\\ \tt — r in position 2 (new letter)\\ \tt — i in position 3 (new letter)\\ \tt — a in position 4 (as required)\\ \tt — l in position 5 (different from pos1 in "lucky" and pos3 in "salad", fitting the pattern "???al")\\ \tt No contradictions. That's good, so let's go with "trial".\\ \tt </cot>\\ \tt guess: trial\\ \tt \\ \tt -----------\\ \tt Let's begin with your first guess. 
	  } 
	   } 
	   } 
	 & \\ 
 

    \theutterance \stepcounter{utterance}  

    & \multicolumn{4}{p{0.6\linewidth}}{\cellcolor[rgb]{0.8,1,0.9}{%
	\makecell[{{p{\linewidth}}}]{% 
	  \tt {\tiny [A$\rangle$GM]}  
	 <cot>\\ \tt I'm starting with a common five-letter English word, which includes frequent consonants and vowels.\\ \tt </cot>\\ \tt guess: house 
	  } 
	   } 
	   } 
	 & & \\ 
 

    \theutterance \stepcounter{utterance}  

    & & & \multicolumn{2}{p{0.3\linewidth}}{\cellcolor[rgb]{0.95,0.95,0.95}{%
	\makecell[{{p{\linewidth}}}]{% 
	  \tt {\tiny [GM$|$GM]}  
	 attempts = 1\\ \tt target_word = bench\\ \tt guess = house\\ \tt guess_feedback = h<yellow> o<red> u<red> s<red> e<yellow> 
	  } 
	   } 
	   } 
	 & & \\ 
 

    \theutterance \stepcounter{utterance}  

    & & \multicolumn{4}{p{0.6\linewidth}}{\cellcolor[rgb]{0.9,0.9,0.9}{%
	\makecell[{{p{\linewidth}}}]{% 
	  \tt {\tiny [A$\langle$GM]}  
	 guess_feedback: h<yellow> o<red> u<red> s<red> e<yellow> 
	  } 
	   } 
	   } 
	 & \\ 
 

    \theutterance \stepcounter{utterance}  

    & \multicolumn{4}{p{0.6\linewidth}}{\cellcolor[rgb]{0.8,1,0.9}{%
	\makecell[{{p{\linewidth}}}]{% 
	  \tt {\tiny [A$\rangle$GM]}  
	 <cot>\\ \tt 1. Feedback explanation\\ \tt The letters 'h' and 'e' are in the word but not in positions 1 and 5, respectively. Also, 'o', 'u', and 's' are not in the word (red).\\ \tt 2. Structured representation\\ \tt keep = {}  # no green letters yet\\ \tt try_in_position = {'h': {2, 3, 4, 5}, 'e': {1, 2, 3, 4}}\\ \tt avoid = {'o', 'u', 's'}\\ \tt 3. A fitting 5-letter word\\ \tt Let me think of a valid 5-letter word that fits all our constraints: So I need to avoid {'o', 'u', 's'} and introduce three new letters instead (for example, 't', 'r', 'l', 'f', 'c', 'n', 'p', 'm', 'b'). Also, I'm trying to find a new arrangement for the yellow letters 'h' and 'e'. Maybe try them in different positions? "he" could actually be a substring, possibly followed by two new letters and then by another letter — as in "he__". "heath" comes to mind, but 'h' is in position 1, which is not in try_in_position['h']. However, "the" or "them" or "then" could be substrings of the target word. How about "theme" (T-H-E-M-E): \\ \tt — t in position 1 (new letter)\\ \tt — h in position 2 (different from pos1 in "house", fitting the pattern)\\ \tt — e in position 3 (different from pos5 in "house", fitting the pattern)\\ \tt — m in position 4 (new letter)\\ \tt — e in position 5 (different from pos5 in "house", but this 'e' is not the same 'e' we are looking for, however "theme" still seems like a reasonable guess given the constraints)\\ \tt No contradictions. That's good, so let's go with "theme".\\ \tt </cot>\\ \tt guess: theme 
	  } 
	   } 
	   } 
	 & & \\ 
 

    \theutterance \stepcounter{utterance}  

    & & & \multicolumn{2}{p{0.3\linewidth}}{\cellcolor[rgb]{0.95,0.95,0.95}{%
	\makecell[{{p{\linewidth}}}]{% 
	  \tt {\tiny [GM$|$GM]}  
	 attempts = 2\\ \tt target_word = bench\\ \tt guess = theme\\ \tt guess_feedback = t<red> h<yellow> e<yellow> m<red> e<red> 
	  } 
	   } 
	   } 
	 & & \\ 
 

    \theutterance \stepcounter{utterance}  

    & & \multicolumn{4}{p{0.6\linewidth}}{\cellcolor[rgb]{0.9,0.9,0.9}{%
	\makecell[{{p{\linewidth}}}]{% 
	  \tt {\tiny [A$\langle$GM]}  
	 guess_feedback: t<red> h<yellow> e<yellow> m<red> e<red> 
	  } 
	   } 
	   } 
	 & \\ 
 

    \theutterance \stepcounter{utterance}  

    & \multicolumn{4}{p{0.6\linewidth}}{\cellcolor[rgb]{0.8,1,0.9}{%
	\makecell[{{p{\linewidth}}}]{% 
	  \tt {\tiny [A$\rangle$GM]}  
	 <cot>\\ \tt 1. Feedback explanation\\ \tt The letters 'h' and 'e' are in the word but not in positions 2 and 3, respectively. Also, 't', 'm', and the second 'e' are not in the word (red). Note that the first 'e' in the guess "theme" is yellow, but the second 'e' is red, which implies that there is only one 'e' in the target word.\\ \tt 2. Structured representation\\ \tt keep = {}  # no green letters yet\\ \tt try_in_position = {'h': {1, 4, 5}, 'e': {1, 4, 5}}  # positions 2 and 3 are ruled out for 'h' and 'e'\\ \tt avoid = {'o', 'u', 's', 't', 'm'}\\ \tt 3. A fitting 5-letter word\\ \tt Let me think of a valid 5-letter word that fits all our constraints: So I need to avoid {'o', 'u', 's', 't', 'm'} and introduce three new letters instead (for example, 'r', 'l', 'f', 'c', 'n', 'p', 'b'). Also, I'm trying to find a new arrangement for the yellow letters 'h' and 'e'. Maybe try them in different positions? "he" could actually be a substring, possibly preceded by two new letters — as in "__he". How about "cheer" doesn't work because 'e' is repeated, but "chew" or "che" are not valid guesses since they are not 5-letter words. However "chere" is not a word, but "chear" is not a word either. How about "chevy" doesn't work, but "chelp" or "cherm" are not valid guesses since they don't seem like common words and 'm' is in the avoid set. Let's try "chile" or "chink" or "chill" — no, 'l' is not a bad choice, but let's try a different word: "chime" doesn't work because 'm' is in the avoid set and 'i' is a new letter, but we can try a word that has 'h' and 'e' and also 'r' and another new letter 'l' or 'c' or 'n' or 'p' or 'b' or 'f': How about "chelf" doesn't seem like a word, but "whel" is not a word, however "whel" has an 'h' and an 'e' and also 'w'. Let's think of a word that has 'w', 'h', 'e'. How about "whe" as a substring: "wheat" or "wheel" or "wheft" — no, 't' is in the avoid set and 'f' could be a good choice, but let's try "whe" with a different last two letters: "whelp" doesn't seem like a bad guess, given the constraints.\\ \tt </cot>\\ \tt guess: whelp 
	  } 
	   } 
	   } 
	 & & \\ 
 

    \theutterance \stepcounter{utterance}  

    & & & \multicolumn{2}{p{0.3\linewidth}}{\cellcolor[rgb]{0.95,0.95,0.95}{%
	\makecell[{{p{\linewidth}}}]{% 
	  \tt {\tiny [GM$|$GM]}  
	 attempts = 3\\ \tt target_word = bench\\ \tt guess = whelp\\ \tt guess_feedback = w<red> h<yellow> e<yellow> l<red> p<red> 
	  } 
	   } 
	   } 
	 & & \\ 
 

    \theutterance \stepcounter{utterance}  

    & & \multicolumn{4}{p{0.6\linewidth}}{\cellcolor[rgb]{0.9,0.9,0.9}{%
	\makecell[{{p{\linewidth}}}]{% 
	  \tt {\tiny [A$\langle$GM]}  
	 guess_feedback: w<red> h<yellow> e<yellow> l<red> p<red> 
	  } 
	   } 
	   } 
	 & \\ 
 

    \theutterance \stepcounter{utterance}  

    & \multicolumn{4}{p{0.6\linewidth}}{\cellcolor[rgb]{0.8,1,0.9}{%
	\makecell[{{p{\linewidth}}}]{% 
	  \tt {\tiny [A$\rangle$GM]}  
	 <cot>\\ \tt 1. Feedback explanation\\ \tt The letters 'h' and 'e' are in the word but not in positions 2 and 3, respectively. Also, 'w', 'l', and 'p' are not in the word (red).\\ \tt 2. Structured representation\\ \tt keep = {}  # no green letters yet\\ \tt try_in_position = {'h': {1, 4, 5}, 'e': {1, 4, 5}}  # positions 2 and 3 are ruled out for 'h' and 'e'\\ \tt avoid = {'o', 'u', 's', 't', 'm', 'w', 'l', 'p'}\\ \tt 3. A fitting 5-letter word\\ \tt Let me think of a valid 5-letter word that fits all our constraints: So I need to avoid {'o', 'u', 's', 't', 'm', 'w', 'l', 'p'} and introduce three new letters instead (for example, 'r', 'c', 'n', 'b', 'f', 'd', 'k', 'y'). Also, I'm trying to find a new arrangement for the yellow letters 'h' and 'e'. Maybe try them in different positions? "he" could actually be a substring, possibly preceded by two new letters — as in "__he". How about "chew" is not a valid guess since 'w' is in the avoid set, but let's try a word that has 'h' and 'e' and also 'r' and another new letter 'c' or 'n' or 'b' or 'f' or 'd' or 'k' or 'y' or 'g' or 'j' or 'x' or 'v' or 'z' or 'q' or 'i': How about "chey" or "chee" — no, 'e' is repeated. Let's think of a word that has 'h' and 'e'. How about "they" (T-H-E-Y): \\ \tt — t in position 1 (new letter)\\ \tt — h in position 2 (different from pos1 in previous guesses, fitting the pattern)\\ \tt — e in position 3 (different from pos3 in previous guesses, fitting the pattern)\\ \tt — y in position 4 (new letter)\\ \tt — ? in position 5 (new letter)\\ \tt Let's fill in the last position with a letter that is not in the avoid set. How about "they" with a different last letter: "thea" doesn't seem like a word, but let's try "the" with a different last two letters: How about "ther" and then add a letter that is not in the avoid set: "thery" doesn't seem like a word, but let's try "ther" and then add a 'b': "therb" is not a word, however "therm" is not a valid guess since 'm' is in the avoid set. Let's try "ther" and then add a 'c': "therc" is not a word, however "there" seems like a reasonable guess given the constraints.\\ \tt </cot>\\ \tt guess: there 
	  } 
	   } 
	   } 
	 & & \\ 
 

    \theutterance \stepcounter{utterance}  

    & & & \multicolumn{2}{p{0.3\linewidth}}{\cellcolor[rgb]{0.95,0.95,0.95}{%
	\makecell[{{p{\linewidth}}}]{% 
	  \tt {\tiny [GM$|$GM]}  
	 attempts = 4\\ \tt target_word = bench\\ \tt guess = there\\ \tt guess_feedback = t<red> h<yellow> e<yellow> r<red> e<red> 
	  } 
	   } 
	   } 
	 & & \\ 
 

    \theutterance \stepcounter{utterance}  

    & & \multicolumn{4}{p{0.6\linewidth}}{\cellcolor[rgb]{0.9,0.9,0.9}{%
	\makecell[{{p{\linewidth}}}]{% 
	  \tt {\tiny [A$\langle$GM]}  
	 guess_feedback: t<red> h<yellow> e<yellow> r<red> e<red> 
	  } 
	   } 
	   } 
	 & \\ 
 

    \theutterance \stepcounter{utterance}  

    & \multicolumn{4}{p{0.6\linewidth}}{\cellcolor[rgb]{0.8,1,0.9}{%
	\makecell[{{p{\linewidth}}}]{% 
	  \tt {\tiny [A$\rangle$GM]}  
	 <cot>\\ \tt 1. Feedback explanation\\ \tt The letters 'h' and 'e' are in the word but not in positions 2 and 3, respectively. Also, 't', 'r', and the second 'e' are not in the word (red). Note that the first 'e' in the guess "there" is yellow, but the second 'e' is red, which implies that there is only one 'e' in the target word.\\ \tt 2. Structured representation\\ \tt keep = {}  # no green letters yet\\ \tt try_in_position = {'h': {1, 4, 5}, 'e': {1, 4, 5}}  # positions 2 and 3 are ruled out for 'h' and 'e'\\ \tt avoid = {'o', 'u', 's', 't', 'm', 'w', 'l', 'p', 'r'}\\ \tt 3. A fitting 5-letter word\\ \tt Let me think of a valid 5-letter word that fits all our constraints: So I need to avoid {'o', 'u', 's', 't', 'm', 'w', 'l', 'p', 'r'} and introduce three new letters instead (for example, 'c', 'n', 'b', 'f', 'd', 'k', 'y', 'g', 'j', 'x', 'v', 'z', 'q', 'i'). Also, I'm trying to find a new arrangement for the yellow letters 'h' and 'e'. Maybe try them in different positions? "he" could actually be a substring, possibly preceded by two new letters — as in "__he". How about "chew" is not a valid guess since 'w' is in the avoid set, but let's try a word that has 'h' and 'e' and also 'n' and another new letter 'c' or 'b' or 'f' or 'd' or 'k' or 'y' or 'g' or 'j' or 'x' or 'v' or 'z' or 'q' or 'i': How about "chey" or "chee" — no, 'e' is repeated. Let's think of a word that has 'h' and 'e'. How about "he" in positions 4 and 5: "cheh" is not a word, however "chee" is not a valid guess since 'e' is repeated. Let's try "he" in positions 4 and 5 with a different first three letters: How about "ghe" or "bhe" — no, these are not words. Let's try a word that has 'h' in position 5 and 'e' in position 4: How about "bhe" or "ghe" — no, these are not words. However, let's try a word that has 'h' in position 5 and 'e' in position 4 with a different first three letters: How about "ache" or "bche" — no, 'bche' is not a word, but let's try "ache" with a different first letter: How about "nche" — no, this is not a word. Let's try "ache" with the letters we have: How about "ache" seems like a reasonable guess given the constraints.\\ \tt </cot>\\ \tt guess: ache 
	  } 
	   } 
	   } 
	 & & \\ 
 

    \theutterance \stepcounter{utterance}  

    & & & \multicolumn{2}{p{0.3\linewidth}}{\cellcolor[rgb]{0.95,0.95,0.95}{%
	\makecell[{{p{\linewidth}}}]{% 
	  \tt {\tiny [GM$|$GM]}  
	 Guess does not conform to the format rules\\ \tt Error: The length of the guessed word is not 5. 
	  } 
	   } 
	   } 
	 & & \\ 
 

    \theutterance \stepcounter{utterance}  

    & & & \multicolumn{2}{p{0.3\linewidth}}{\cellcolor[rgb]{0.95,0.95,0.95}{%
	\makecell[{{p{\linewidth}}}]{% 
	  \tt {\tiny [GM$|$GM]}  
	 game_result = ABORT 
	  } 
	   } 
	   } 
	 & & \\ 
 

\end{supertabular}
}

\end{document}
