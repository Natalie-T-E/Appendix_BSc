
\documentclass{article}
\usepackage{colortbl}
\usepackage{makecell}
\usepackage{multirow}
\usepackage{supertabular}

\begin{document}

\newcounter{utterance}

\twocolumn

{ \footnotesize  \setcounter{utterance}{1}
\setlength{\tabcolsep}{0pt}
\begin{supertabular}{c@{$\;$}|p{.15\linewidth}@{}p{.15\linewidth}p{.15\linewidth}p{.15\linewidth}p{.15\linewidth}p{.15\linewidth}}

    \# & $\;$A & \multicolumn{4}{c}{Game Master} & $\;\:$B\\
    \hline 

    \theutterance \stepcounter{utterance}  

    & & \multicolumn{4}{p{0.6\linewidth}}{\cellcolor[rgb]{0.9,0.9,0.9}{%
	\makecell[{{p{\linewidth}}}]{% 
	  \tt {\tiny [A$\langle$GM]}  
	 You are a language wizard who likes to guess words by using the given rules.\\ \tt \\ \tt Welcome to Wordle! You have six attempts to guess the target word, a valid English word of five lowercase letters (a-z).\\ \tt After each guess, your answer will be validated, and you will receive feedback indicating which letters are correct (green), which letters are correct but in the wrong position (yellow), and which letters are incorrect (red). Utilize this feedback when determining which letters to include or exclude in your next guess.\\ \tt \\ \tt Please follow these instructions rigorously:\\ \tt \\ \tt In any case, the first token of your answer must be the tag "<cot>". The demanded reasoning steps to be enclosed in <cot></cot> depend on where you are in the game.\\ \tt \\ \tt For the starting guess:\\ \tt There is only one step within <cot></cot>: Provide a short explanation for your guess.\\ \tt Then state your guess, using the tag "guess:".\\ \tt \\ \tt For all subsequent guesses:\\ \tt Please adhere to the following 3 reasoning steps within <cot></cot>:\\ \tt 1) Explain the provided feedback. Be specific about both letters and positions (1-5). It can be helpful to mention relevant information from your earlier guess feedbacks as well.\\ \tt 2) Write a structured representation of what you know so far about the target word. Define a variable "keep" if you already encountered any green letters, and map these letters to their known positions. If there is only one remaining possible position for a yellow letter, it should also be included in the variable "keep". Define a variable "try_in_position" to formalize your conclusion regarding letters you know are in the target word but not yet where (when there is more than one position left to try), mapping these letters to their possible positions. Represent the set of eliminated letters by defining a variable "avoid". In addition, you may also define possible patterns. You are encouraged to use Python syntax in this step.\\ \tt 3) Find your next guess: a valid 5-letter word in accordance with the identified constraints. Important: Take your first idea that does not contradict any constraints. Do not forget the word length constraint. If your guess consists of more or less than five letters, the game will be aborted! Be concise; you have a limited number of tokens! You must finish this step with the closing tag </cot>.\\ \tt Then state the guess you chose in step 3 once more, using the tag "guess:". Please do NOT come up with a new guess at this point.\\ \tt \\ \tt Do not generate any additional text.\\ \tt \\ \tt -----------\\ \tt Here is an example for a starting guess:\\ \tt Prompt: … Let's begin with your first guess.\\ \tt \\ \tt Example Answer:\\ \tt <cot>\\ \tt I'm starting with a common five-letter English word, which includes frequent consonants and vowels.\\ \tt </cot>\\ \tt guess: house\\ \tt \\ \tt -----------\\ \tt Here are two examples for a second/third/fourth/fifth/sixth guess:\\ \tt \\ \tt Example I:\\ \tt Context:\\ \tt 1st guess: "crane", guess_feedback: c<red> r<yellow> a<red> n<yellow> e<yellow>\\ \tt \\ \tt Answer:\\ \tt <cot>\\ \tt 1. Feedback explanation\\ \tt The letters 'r', 'n' and 'e' are in the word but not in positions 2, 4, 5, respectively. Also, 'c' and 'a' are not in the word (red).\\ \tt 2. Structured representation\\ \tt keep = {}  # no green letters yet\\ \tt try_in_position = {'r': {1, 3, 4, 5}, 'n': {1, 2, 3, 5}, 'e': {1, 2, 3, 4}}\\ \tt avoid = {'c', 'a'}\\ \tt 3. A fitting 5-letter word\\ \tt Let me think of a valid 5-letter word that fits all our constraints: So I need to avoid {'c', 'a'} and introduce two new letters instead (for example, 's', 't', 'l', 'f', 'o', 'u', 'i' or 'y'). Also, I'm trying to find a new arrangement for the yellow letters 'r', 'n' and 'e'. Maybe try them in reverse order? "en" could actually be a substring, possibly followed by a new letter and then by 'r' — as in "_en_r" (alternatively, "en_r_"). "tenor" comes to mind. Let me check the constraints: 'e' is now placed in position 2, which is indeed in try_in_position['e'] (fine), 'n' is moved to position 3, which is in try_in_position['n'] (fine), 'r' is now in position 5, which is in try_in_position['r'] (fine). "tenor" doesn't include any eliminated letters. All criteria are fulfilled, so "tenor" is a good next guess.\\ \tt </cot>\\ \tt guess: tenor\\ \tt \\ \tt Example II:\\ \tt Context:\\ \tt 1st guess: "lucky", guess_feedback: l<yellow> u<red> c<red> k<red> y<red>\\ \tt 2nd guess: "salad", guess_feedback: s<red> a<red> l<yellow> a<green> d<red>\\ \tt \\ \tt Answer:\\ \tt <cot>\\ \tt 1. Feedback explanation\\ \tt The feedback indicates that 'a' is correct in position 4 (and not in position 2). The letter 'l' is present but not in position 1 (yellow in "lucky") or 3 (yellow in "salad"). Moreover, position 4 is ruled out implicitly via the green 'a', this means 'l' must be in position 2 or 5. In addition to the previously eliminated letters {'u', 'c', 'k', 'y'}, 's' and 'd' are not in the word at all (red).\\ \tt 2. Structured representation\\ \tt keep = {'a': 4}\\ \tt try_in_position = {'l': {2, 5}}\\ \tt avoid = {'u', 'c', 'k', 'y', 's', 'd'}\\ \tt patterns = [\\ \tt     "_l_a_",  # 'l' in position 2\\ \tt     "???al",  # 'l' in position 5\\ \tt ]\\ \tt 3. A fitting 5-letter word\\ \tt "???al" (patterns[1]) seems like a good starting point. To fill in the placeholders, I need three new letters. Common consonants I haven't tried yet include: 't', 'r', 'n'. How about "trial" (T-R-I-A-L):\\ \tt — t in position 1 (new letter)\\ \tt — r in position 2 (new letter)\\ \tt — i in position 3 (new letter)\\ \tt — a in position 4 (as required)\\ \tt — l in position 5 (different from pos1 in "lucky" and pos3 in "salad", fitting the pattern "???al")\\ \tt No contradictions. That's good, so let's go with "trial".\\ \tt </cot>\\ \tt guess: trial\\ \tt \\ \tt -----------\\ \tt Let's begin with your first guess. 
	  } 
	   } 
	   } 
	 & \\ 
 

    \theutterance \stepcounter{utterance}  

    & \multicolumn{4}{p{0.6\linewidth}}{\cellcolor[rgb]{0.8,1,0.9}{%
	\makecell[{{p{\linewidth}}}]{% 
	  \tt {\tiny [A$\rangle$GM]}  
	 <cot>\\ \tt I'm starting with a common five-letter English word, which includes frequent consonants and vowels.\\ \tt </cot>\\ \tt guess: house 
	  } 
	   } 
	   } 
	 & & \\ 
 

    \theutterance \stepcounter{utterance}  

    & & & \multicolumn{2}{p{0.3\linewidth}}{\cellcolor[rgb]{0.95,0.95,0.95}{%
	\makecell[{{p{\linewidth}}}]{% 
	  \tt {\tiny [GM$|$GM]}  
	 attempts = 1\\ \tt target_word = until\\ \tt guess = house\\ \tt guess_feedback = h<red> o<red> u<yellow> s<red> e<red> 
	  } 
	   } 
	   } 
	 & & \\ 
 

    \theutterance \stepcounter{utterance}  

    & & \multicolumn{4}{p{0.6\linewidth}}{\cellcolor[rgb]{0.9,0.9,0.9}{%
	\makecell[{{p{\linewidth}}}]{% 
	  \tt {\tiny [A$\langle$GM]}  
	 guess_feedback: h<red> o<red> u<yellow> s<red> e<red> 
	  } 
	   } 
	   } 
	 & \\ 
 

    \theutterance \stepcounter{utterance}  

    & \multicolumn{4}{p{0.6\linewidth}}{\cellcolor[rgb]{0.8,1,0.9}{%
	\makecell[{{p{\linewidth}}}]{% 
	  \tt {\tiny [A$\rangle$GM]}  
	 <cot>\\ \tt 1. Feedback explanation\\ \tt The letter 'u' is in the word but not in position 3. The letters 'h', 'o', 's', and 'e' are not in the word.\\ \tt 2. Structured representation\\ \tt keep = {}  # no green letters yet\\ \tt try_in_position = {'u': {1, 2, 4, 5}}\\ \tt avoid = {'h', 'o', 's', 'e'}\\ \tt 3. A fitting 5-letter word\\ \tt Let me think of a valid 5-letter word that fits all our constraints: So I need to avoid {'h', 'o', 's', 'e'} and introduce four new letters. How about "bulky" or "bungy" — no, these aren't valid options since they have more than one 'u' or other issues, but "bully" could work if I replace the 'y' with another letter to get a word like "bulks" — no, 's' is eliminated. Another option could be "bucks" — no, 's' is eliminated. What about "bunks" — no, 's' is eliminated. Let's try "bunch" — no, 'h' is eliminated. How about "lunch" or "lunck" — no, these aren't valid options, but "lunch" is close. Let's think of other words containing 'u' and a common set of letters. Maybe "unzip" or "unify" — no, these have a 'y' or other issues, but "unit" or "unis" could work if I replace the last letter with another one to get a word like "units" — no, 's' is eliminated. Another option could be "union" — no, 'o' is eliminated and 'n' isn't in try_in_position. What about "urns" — no, 's' is eliminated. Let me think of a valid 5-letter word that fits all our constraints and includes 'u' in a different position. How about "tunic" — no, 'i' and 'c' could work, but let's check the constraints: 'u' is now placed in position 2, which is in try_in_position['u'] (fine), and the word doesn't include any eliminated letters. All criteria are fulfilled, so "tunic" is a good next guess.\\ \tt </cot>\\ \tt guess: tunic 
	  } 
	   } 
	   } 
	 & & \\ 
 

    \theutterance \stepcounter{utterance}  

    & & & \multicolumn{2}{p{0.3\linewidth}}{\cellcolor[rgb]{0.95,0.95,0.95}{%
	\makecell[{{p{\linewidth}}}]{% 
	  \tt {\tiny [GM$|$GM]}  
	 attempts = 2\\ \tt target_word = until\\ \tt guess = tunic\\ \tt guess_feedback = t<yellow> u<yellow> n<yellow> i<green> c<red> 
	  } 
	   } 
	   } 
	 & & \\ 
 

    \theutterance \stepcounter{utterance}  

    & & \multicolumn{4}{p{0.6\linewidth}}{\cellcolor[rgb]{0.9,0.9,0.9}{%
	\makecell[{{p{\linewidth}}}]{% 
	  \tt {\tiny [A$\langle$GM]}  
	 guess_feedback: t<yellow> u<yellow> n<yellow> i<green> c<red> 
	  } 
	   } 
	   } 
	 & \\ 
 

    \theutterance \stepcounter{utterance}  

    & \multicolumn{4}{p{0.6\linewidth}}{\cellcolor[rgb]{0.8,1,0.9}{%
	\makecell[{{p{\linewidth}}}]{% 
	  \tt {\tiny [A$\rangle$GM]}  
	 <cot>\\ \tt 1. Feedback explanation\\ \tt The letters 't', 'u', and 'n' are in the word but not in positions 1, 2, and 3, respectively. The letter 'i' is correct and in position 5. The letter 'c' is not in the word.\\ \tt 2. Structured representation\\ \tt keep = {'i': 5}\\ \tt try_in_position = {'t': {2, 3, 4}, 'u': {1, 3, 4}, 'n': {1, 2, 4}}\\ \tt avoid = {'h', 'o', 's', 'e', 'c'}\\ \tt 3. A fitting 5-letter word\\ \tt Let me think of a valid 5-letter word that fits all our constraints: So I need to avoid {'h', 'o', 's', 'e', 'c'} and introduce a new arrangement for the yellow letters 't', 'u', and 'n'. Maybe try them in a different order? How about "tunis" — no, 's' is eliminated. What about "turni" — no, it's not a valid word, but "turn" could work if I add a new letter to get a word like "turns" — no, 's' is eliminated. Let's think of other words containing 't', 'u', 'n', and 'i'. Maybe "unit" or "unis" — no, these aren't valid options since they have 's' or other issues, but "unint" isn't a word. How about "tunic" was close, let's change the 'c' to another letter to get a word like "tunis" — no, 's' is eliminated. Let me think of a valid 5-letter word that fits all our constraints and includes 't', 'u', 'n', and 'i' in different positions. How about "nutri" — yes, it's a valid word: 'n' in position 1, 'u' in position 2, 't' in position 3, 'r' in position 4, and 'i' in position 5. Let me check the constraints: 'i' is now placed in position 5, which is in keep['i'] (fine), 'n' is now in position 1, which is in try_in_position['n'] (fine), 'u' is now in position 2, which is in try_in_position['u'] (fine), 't' is now in position 3, which is in try_in_position['t'] (fine), and the word doesn't include any eliminated letters. All criteria are fulfilled, so "nutri" is a good next guess.\\ \tt </cot>\\ \tt guess: nutri 
	  } 
	   } 
	   } 
	 & & \\ 
 

    \theutterance \stepcounter{utterance}  

    & & & \multicolumn{2}{p{0.3\linewidth}}{\cellcolor[rgb]{0.95,0.95,0.95}{%
	\makecell[{{p{\linewidth}}}]{% 
	  \tt {\tiny [GM$|$GM]}  
	 Guesser Error: NOT_VALID_WORD_FOR_GAME while parsing Player 1's (model: llama-3.3-70b) response 
	  } 
	   } 
	   } 
	 & & \\ 
 

    \theutterance \stepcounter{utterance}  

    & & \multicolumn{4}{p{0.6\linewidth}}{\cellcolor[rgb]{0.9,0.9,0.9}{%
	\makecell[{{p{\linewidth}}}]{% 
	  \tt {\tiny [A$\langle$GM]}  
	 Your guess is not a valid word for this game. Please try again. 
	  } 
	   } 
	   } 
	 & \\ 
 

    \theutterance \stepcounter{utterance}  

    & & \multicolumn{4}{p{0.6\linewidth}}{\cellcolor[rgb]{0.9,0.9,0.9}{%
	\makecell[{{p{\linewidth}}}]{% 
	  \tt {\tiny [A$\langle$GM]}  
	 Your guess is not a valid word for this game. Please try again. 
	  } 
	   } 
	   } 
	 & \\ 
 

    \theutterance \stepcounter{utterance}  

    & \multicolumn{4}{p{0.6\linewidth}}{\cellcolor[rgb]{0.8,1,0.9}{%
	\makecell[{{p{\linewidth}}}]{% 
	  \tt {\tiny [A$\rangle$GM]}  
	 <cot>\\ \tt 1. Feedback explanation\\ \tt The letters 't', 'u', and 'n' are in the word but not in positions 1, 2, and 3, respectively. The letter 'i' is correct and in position 5. The letter 'c' is not in the word.\\ \tt 2. Structured representation\\ \tt keep = {'i': 5}\\ \tt try_in_position = {'t': {2, 3, 4}, 'u': {1, 3, 4}, 'n': {1, 2, 4}}\\ \tt avoid = {'h', 'o', 's', 'e', 'c'}\\ \tt 3. A fitting 5-letter word\\ \tt Let me think of a valid 5-letter word that fits all our constraints: So I need to avoid {'h', 'o', 's', 'e', 'c'} and introduce a new arrangement for the yellow letters 't', 'u', and 'n'. Maybe try them in a different order? How about "untie" — no, it's not a bad option, but let's check the constraints: 'u' is now placed in position 1, which is in try_in_position['u'] (fine), 'n' is now in position 2, which is in try_in_position['n'] (fine), 't' is now in position 3, which is in try_in_position['t'] (fine), 'i' is now in position 5, which is in keep['i'] (fine), and 'e' is in position 4, which is not ideal since 'e' is eliminated, so let's try another word. What about "unfit" — no, 'f' could work, but let's check the constraints: 'u' is now placed in position 1, which is in try_in_position['u'] (fine), 'n' is now in position 2, which is in try_in_position['n'] (fine), 'f' is a new letter in position 3, 'i' is now in position 5, which is in keep['i'] (fine), and 't' is now in position 4, which is in try_in_position['t'] (fine), and the word doesn't include any eliminated letters. All criteria are fulfilled, so "unfit" is a good next guess.\\ \tt </cot>\\ \tt guess: unfit 
	  } 
	   } 
	   } 
	 & & \\ 
 

    \theutterance \stepcounter{utterance}  

    & & & \multicolumn{2}{p{0.3\linewidth}}{\cellcolor[rgb]{0.95,0.95,0.95}{%
	\makecell[{{p{\linewidth}}}]{% 
	  \tt {\tiny [GM$|$GM]}  
	 attempts = 3\\ \tt target_word = until\\ \tt guess = unfit\\ \tt guess_feedback = u<green> n<green> f<red> i<green> t<yellow> 
	  } 
	   } 
	   } 
	 & & \\ 
 

    \theutterance \stepcounter{utterance}  

    & & \multicolumn{4}{p{0.6\linewidth}}{\cellcolor[rgb]{0.9,0.9,0.9}{%
	\makecell[{{p{\linewidth}}}]{% 
	  \tt {\tiny [A$\langle$GM]}  
	 guess_feedback: u<green> n<green> f<red> i<green> t<yellow> 
	  } 
	   } 
	   } 
	 & \\ 
 

    \theutterance \stepcounter{utterance}  

    & \multicolumn{4}{p{0.6\linewidth}}{\cellcolor[rgb]{0.8,1,0.9}{%
	\makecell[{{p{\linewidth}}}]{% 
	  \tt {\tiny [A$\rangle$GM]}  
	 <cot>\\ \tt 1. Feedback explanation\\ \tt The letters 'u' and 'n' are correct and in positions 1 and 2, respectively. The letter 'i' is also correct and in position 5. The letter 't' is in the word but not in position 5. The letter 'f' is not in the word.\\ \tt 2. Structured representation\\ \tt keep = {'u': 1, 'n': 2, 'i': 5}\\ \tt try_in_position = {'t': {3, 4}}\\ \tt avoid = {'h', 'o', 's', 'e', 'c', 'f'}\\ \tt 3. A fitting 5-letter word\\ \tt Let me think of a valid 5-letter word that fits all our constraints: So I need to avoid {'h', 'o', 's', 'e', 'c', 'f'} and introduce 't' in either position 3 or 4. How about "unit" — no, it's not a valid option since 't' is in the wrong position. What about "unite" — no, 'e' is eliminated. Let me think of a valid 5-letter word that fits all our constraints and includes 'u', 'n', 't', and 'i' in the correct positions. How about "until" — yes, it's a valid word: 'u' in position 1, 'n' in position 2, 't' in position 3, 'i' in position 5, and 'l' in position 4. Let me check the constraints: 'u' is now placed in position 1, which is in keep['u'] (fine), 'n' is now in position 2, which is in keep['n'] (fine), 't' is now in position 3, which is in try_in_position['t'] (fine), 'i' is now in position 5, which is in keep['i'] (fine), and the word doesn't include any eliminated letters. All criteria are fulfilled, so "until" is a good next guess.\\ \tt </cot>\\ \tt guess: until 
	  } 
	   } 
	   } 
	 & & \\ 
 

    \theutterance \stepcounter{utterance}  

    & & & \multicolumn{2}{p{0.3\linewidth}}{\cellcolor[rgb]{0.95,0.95,0.95}{%
	\makecell[{{p{\linewidth}}}]{% 
	  \tt {\tiny [GM$|$GM]}  
	 attempts = 4\\ \tt target_word = until\\ \tt guess = until\\ \tt guess_feedback = u<green> n<green> t<green> i<green> l<green> 
	  } 
	   } 
	   } 
	 & & \\ 
 

    \theutterance \stepcounter{utterance}  

    & & & \multicolumn{2}{p{0.3\linewidth}}{\cellcolor[rgb]{0.95,0.95,0.95}{%
	\makecell[{{p{\linewidth}}}]{% 
	  \tt {\tiny [GM$|$GM]}  
	 game_result = WIN 
	  } 
	   } 
	   } 
	 & & \\ 
 

\end{supertabular}
}

\end{document}
