
\documentclass{article}
\usepackage{colortbl}
\usepackage{makecell}
\usepackage{multirow}
\usepackage{supertabular}

\begin{document}

\newcounter{utterance}

\twocolumn

{ \footnotesize  \setcounter{utterance}{1}
\setlength{\tabcolsep}{0pt}
\begin{supertabular}{c@{$\;$}|p{.15\linewidth}@{}p{.15\linewidth}p{.15\linewidth}p{.15\linewidth}p{.15\linewidth}p{.15\linewidth}}

    \# & $\;$A & \multicolumn{4}{c}{Game Master} & $\;\:$B\\
    \hline 

    \theutterance \stepcounter{utterance}  

    & & \multicolumn{4}{p{0.6\linewidth}}{\cellcolor[rgb]{0.9,0.9,0.9}{%
	\makecell[{{p{\linewidth}}}]{% 
	  \tt {\tiny [A$\langle$GM]}  
	 You are a language wizard who likes to guess words by using the given rules.\\ \tt \\ \tt Welcome to Wordle! You have six attempts to guess the target word, a valid English word of five lowercase letters (a-z).\\ \tt After each guess, your answer will be validated, and you will receive feedback indicating which letters are correct (green), which letters are correct but in the wrong position (yellow), and which letters are incorrect (red). Utilize this feedback when determining which letters to include or exclude in your next guess.\\ \tt \\ \tt Please follow these instructions rigorously:\\ \tt \\ \tt In any case, the first token of your answer must be the tag "<cot>". The demanded reasoning steps to be enclosed in <cot></cot> depend on where you are in the game.\\ \tt \\ \tt For the starting guess:\\ \tt There is only one step within <cot></cot>: Provide a short explanation for your guess.\\ \tt Then state your guess, using the tag "guess:".\\ \tt \\ \tt For all subsequent guesses:\\ \tt Please adhere to the following 2 reasoning steps within <cot></cot>:\\ \tt 1) Explain the provided feedback. Be specific about both letters and positions (1-5). It can be helpful to mention relevant information from your earlier guess feedbacks as well.\\ \tt 2) Find your next guess: a valid 5-letter word in accordance with the identified constraints. Important: Take your first idea that does not contradict any constraints. Do not forget the word length constraint. If your guess consists of more or less than five letters, the game will be aborted! Be concise; you have a limited number of tokens! You must finish this step with the closing tag </cot>.\\ \tt Then state the guess you chose in step 2 once more, using the tag "guess:". Please do NOT come up with a new guess at this point.\\ \tt \\ \tt Do not generate any additional text.\\ \tt \\ \tt -----------\\ \tt Here is an example for a starting guess:\\ \tt Prompt: … Let's begin with your first guess.\\ \tt \\ \tt Example Answer:\\ \tt <cot>\\ \tt I'm starting with a common five-letter English word, which includes frequent consonants and vowels.\\ \tt </cot>\\ \tt guess: house\\ \tt \\ \tt -----------\\ \tt Here are two examples for a second/third/fourth/fifth/sixth guess:\\ \tt \\ \tt Example I:\\ \tt Context:\\ \tt 1st guess: "crane", guess_feedback: c<red> r<yellow> a<red> n<yellow> e<yellow>\\ \tt \\ \tt Answer:\\ \tt <cot>\\ \tt 1. Feedback explanation\\ \tt The letters 'r', 'n' and 'e' are in the word but not in positions 2, 4, 5, respectively. Also, 'c' and 'a' are not in the word (red).\\ \tt 2. A fitting 5-letter word\\ \tt Let me think of a valid 5-letter word that fits all our constraints: So I need to avoid 'c' and 'a' and introduce two new letters instead (for example, 's', 't', 'l', 'f', 'o', 'u', 'i' or 'y'). Also, I'm trying to find a new arrangement for the yellow letters 'r', 'n' and 'e'. Maybe try them in reverse order? "en" could actually be a substring, possibly followed by a new letter and then by 'r' — as in "_en_r" (alternatively, "en_r_"). "tenor" comes to mind. Let me check the constraints: 'e' is now placed in position 2, which is indeed different from position 5 in "crane" (fine), 'n' was incorrect in position 4 and is moved to position 3 (fine), 'r' was incorrect in position 2 and is now in position 5 (fine). "tenor" doesn't include any eliminated letters. All criteria are fulfilled, so "tenor" is a good next guess.\\ \tt </cot>\\ \tt guess: tenor\\ \tt \\ \tt Example II:\\ \tt Context:\\ \tt 1st guess: "lucky", guess_feedback: l<yellow> u<red> c<red> k<red> y<red>\\ \tt 2nd guess: "salad", guess_feedback: s<red> a<red> l<yellow> a<green> d<red>\\ \tt \\ \tt Answer:\\ \tt <cot>\\ \tt 1. Feedback explanation\\ \tt The feedback indicates that 'a' is correct in position 4 (and not in position 2). The letter 'l' is present but not in position 1 (yellow in "lucky") or 3 (yellow in "salad"). Moreover, position 4 is ruled out implicitly via the green 'a', this means 'l' must be in position 2 or 5. In addition to the previously eliminated letters ('u', 'c', 'k', 'y'), 's' and 'd' are not in the word at all (red).\\ \tt 2. A fitting 5-letter word\\ \tt I need a word that fits the pattern "_l_a_" with 'l' in position 2, alternatively "???al", placing 'l' in position 5. "???al" seems like a good starting point. To fill in the placeholders, I need three new letters. Common consonants I haven't tried yet include: 't', 'r', 'n'. How about "trial" (T-R-I-A-L):\\ \tt — t in position 1 (new letter)\\ \tt — r in position 2 (new letter)\\ \tt — i in position 3 (new letter)\\ \tt — a in position 4 (as required)\\ \tt — l in position 5 (different from pos1 in "lucky" and pos3 in "salad", fitting the pattern "???al")\\ \tt No contradictions. That's good, so let's go with "trial".\\ \tt </cot>\\ \tt guess: trial\\ \tt \\ \tt -----------\\ \tt Let's begin with your first guess. 
	  } 
	   } 
	   } 
	 & \\ 
 

    \theutterance \stepcounter{utterance}  

    & \multicolumn{4}{p{0.6\linewidth}}{\cellcolor[rgb]{0.8,1,0.9}{%
	\makecell[{{p{\linewidth}}}]{% 
	  \tt {\tiny [A$\rangle$GM]}  
	 <cot>\\ \tt I'm starting with a common five-letter English word, which includes frequent consonants and vowels, such as "house".\\ \tt </cot>\\ \tt guess: house 
	  } 
	   } 
	   } 
	 & & \\ 
 

    \theutterance \stepcounter{utterance}  

    & & & \multicolumn{2}{p{0.3\linewidth}}{\cellcolor[rgb]{0.95,0.95,0.95}{%
	\makecell[{{p{\linewidth}}}]{% 
	  \tt {\tiny [GM$|$GM]}  
	 attempts = 1\\ \tt target_word = today\\ \tt guess = house\\ \tt guess_feedback = h<red> o<green> u<red> s<red> e<red> 
	  } 
	   } 
	   } 
	 & & \\ 
 

    \theutterance \stepcounter{utterance}  

    & & \multicolumn{4}{p{0.6\linewidth}}{\cellcolor[rgb]{0.9,0.9,0.9}{%
	\makecell[{{p{\linewidth}}}]{% 
	  \tt {\tiny [A$\langle$GM]}  
	 guess_feedback: h<red> o<green> u<red> s<red> e<red> 
	  } 
	   } 
	   } 
	 & \\ 
 

    \theutterance \stepcounter{utterance}  

    & \multicolumn{4}{p{0.6\linewidth}}{\cellcolor[rgb]{0.8,1,0.9}{%
	\makecell[{{p{\linewidth}}}]{% 
	  \tt {\tiny [A$\rangle$GM]}  
	 <cot>\\ \tt 1. Feedback explanation\\ \tt The letter 'o' is correct and in the right position (position 2), while 'h', 'u', 's', and 'e' are not in the word (red).\\ \tt 2. A fitting 5-letter word\\ \tt Let me think of a valid 5-letter word that fits the constraint of having 'o' in position 2. A possible word could be "boast" or "coast" but since 'h' and 's' are not allowed, how about "boils" or "fools" or "moons" or "hooks" but 'u' and 's' are not allowed, so "boons" or "looms" or "rooks" but 'h' is not allowed, another option could be "brook" but 'u' is not allowed and 's' is not allowed and another option "cooks" but 'c' and 's' could be but 'h' and 'u' are not allowed and another option could be "flood" but 'u' and 's' and 'd' could be but 'h' is not allowed and another option "gloom" but 'u' is not allowed, another option could be "moods" but 'u' and 's' and 'd' could be but 'h' is not allowed. Let's try "brook" without the 'u' and 's' and 'h', so "brook" becomes "book" but 'u' is not allowed and 's' is not allowed and another option could be "flood" without 'u' so "flood" becomes "flood" without 'u' and 's' so "flood" becomes "flood" without the 'u' so "flood" becomes "flood" without 'u' so "flood" becomes "flood" without the 'u' and 's' and 'h' and 'd' could be but 'u' is not allowed so let's try "flood" without 'u' so "flood" becomes "flood" without the 'u' so "flood" becomes "flood" without 'u' so "flood" becomes "flood" without 'u' and 's' so let's try "flood" without the 'u' so "flood" becomes "flood" without the 'u' so "flood" becomes "flood" without the 'u' so let's try "flood" without the 'u' and 's' so let's try "flood" without the 'u' so let's try "flood" without 'u' so let's try "flood" without 'u' so let's try "flood" without the 'u' so let's try "flood" without 'u' so let's try "flood" without 'u' and 's' so let's try "flood" without 'u' so let's try "flood" without 'u' so let's try "flood" without the 'u' so let's try "flood" without 'u' and 's' so let's try "flood" without 'u' so let's try "flood" without 'u' so let's try "flood" without 'u' so let's try "flood" without 'u' so let's try "flood" without the 'u' so let's try "flood" without 'u' so let's try "flood" without 'u' so let's try "flood" without 'u' and 's' so let's try "flood" without 'u' so let's try "flood" without 'u' so let's try "flood" without 'u' so let's try "flood" without 'u' so let's try "flood" without 'u' so let's try "flood" without 'u' so let's try "flood" without 'u' and 's' so let's try "flood" without 'u' so let's try "flood" without 'u' so let's try "flood" without 'u' so let's try "flood" without 'u' so let's try "flood" without 'u' so let's try "flood" without 'u' so let's try "flood" without 'u' so let's try "flood" without 'u' so let's try "flood" without 'u' so let's try "flood" without 'u' so let's try "flood" without 'u' so let's try "f 
	  } 
	   } 
	   } 
	 & & \\ 
 

    \theutterance \stepcounter{utterance}  

    & & & \multicolumn{2}{p{0.3\linewidth}}{\cellcolor[rgb]{0.95,0.95,0.95}{%
	\makecell[{{p{\linewidth}}}]{% 
	  \tt {\tiny [GM$|$GM]}  
	 Guess does not conform to the format rules\\ \tt Error: The response should always start with the keyword '<cot>' 
	  } 
	   } 
	   } 
	 & & \\ 
 

    \theutterance \stepcounter{utterance}  

    & & & \multicolumn{2}{p{0.3\linewidth}}{\cellcolor[rgb]{0.95,0.95,0.95}{%
	\makecell[{{p{\linewidth}}}]{% 
	  \tt {\tiny [GM$|$GM]}  
	 game_result = ABORT 
	  } 
	   } 
	   } 
	 & & \\ 
 

\end{supertabular}
}

\end{document}
